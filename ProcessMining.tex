\documentclass{article}
\usepackage[utf8]{inputenc}
\usepackage[includeheadfoot, margin=1em,headheight=2em]{geometry}
\usepackage{titling}
\geometry{a4paper, left=2cm, right=2cm, top=2cm, bottom=2cm}
\usepackage{graphicx}
\providecommand{\versionnumber}{1.0.0}
\usepackage{enumitem}
\usepackage{array}
\usepackage[italian]{babel}
\newcolumntype{P}[1]{>{\centering\arraybackslash}p{#1}}
\renewcommand{\arraystretch}{1.5} % Default value: 1
\setlength{\droptitle}{-6em}

%font
\usepackage[defaultfam,tabular,lining]{montserrat}
\usepackage[T1]{fontenc}
\renewcommand*\oldstylenums[1]{{\fontfamily{Montserrat-TOsF}\selectfont #1}}

%custom bold 
\usepackage[outline]{contour}
\usepackage{xcolor}
\newcommand{\custombold}{\contour{black}}

%table colors
\usepackage{color, colortbl}
\definecolor{Blue}{rgb}{0.51,0.68,0.79}
\definecolor{LightBlue}{rgb}{0.82,0.87,0.90}
\definecolor{LighterBlue}{rgb}{0.93,0.95,0.96}

%Header
\usepackage{fancyhdr, xcolor}
\pagestyle{fancy}
\let\oldheadrule\headrule% Copy \headrule into \oldheadrule
\renewcommand{\headrule}{\color{Blue}\oldheadrule}% Add colour to \headrule
\renewcommand{\headrulewidth}{0.2em}
\fancyhead[L]{Studio Process Mining}
\fancyhead[C]{Samuele Vignotto}
\setcounter{secnumdepth}{0}

\title{\Huge{\textbf{Process Mining}}\vspace{-1em}}
\author{Samuele Vignotto}
\date{}
\begin{document}
\maketitle
\begin{figure}[h]
  \centering
  \includegraphics[width=6cm, height=6cm]{Logo/Y_LOGO-SOLO.png}
  \label{fig:immagine}
\end{figure}

\newpage
\tableofcontents
\newpage

\section{I Processi}
Un processo è la serie di passaggi che vengono eseguiti in un ordine particolare per raggiungere un risultato specifico.\\
È nel miglior interesse dell'azienda standardizzare i processi più comuni e ad alto volume. In questo modo si garantisce che le cose vengano fatte nel miglior modo possibile, si rende più facile determinare quando le cose vanno storte e si facilita la capacità di formare nuovi o esistenti dipendenti nel modo migliore per svolgere il loro lavoro.\\
Non importa quanto sia buono un processo, ci sono modi per renderlo migliore.\\

\subsection{Migliorare i processi}
Ecco alcuni dei modi fondamentali per cambiare i processi e renderli migliori:
\begin{itemize}
    \item \custombold{Standardizzazione}: Rendere il processo il più ripetibile possibile e garantire che il processo attuale corrisponda al processo progettato.
    \item \custombold{Snellimento}: Rimuovere dal processo attività rindondanti o non necessarie.
    \item \custombold{Ottimizzazione}: Riprogettare il processo per produrre più valore, come migliorare la qualità o ridurre i costi.
    \item \custombold{Automazione}: Eliminare gli aspetti del processo che richiedono sforzi umani.
\end{itemize}

\section{Process Mining}
Il process mining è un ambito di studio della gestione dei processi aziendali che si occupa dell'analisi dei processi operativi di un'organizzazione basandosi su dati generati dai sistemi informatici. Il suo obiettivo principale è scoprire, monitorare e migliorare i processi reali attraverso l'estrazione e l'analisi dei log di eventi registrati dai sistemi informativi.\\
Ciò che rende possibile il process mining è il fatto che molti dei processi aziendali più diffusi vengono eseguiti tramite sistemi informativi. In senso generale, l'estrazione comporta la ricerca e l'estrazione di qualcosa di valore. Il process mining estrae conoscenze preziose dai log degli eventi prodotti dai sistemi informativi.\\
Per comprendere al meglio il funzionamento del process mining si deve ricordare che un processo è una serie di azioni o passaggi che vanno da un punto di partenza a un punto di arrivo riconosciuto. Questi passaggi possono essere ripetuti e possono essere migliorati nella speranza di passare dal punto di partenza al punto di arrivo nel modo più efficiente e coerente possibile. Quando questi passaggi avvengono in un sistema transazionale, lasciano un'impronta digitale sotto forma di dati dei log degli eventi. Il process mining estrae quei dati e li utilizza per creare un'immagine di come i processi appaiono effettivamente nella pratica.\\
L'aspetto reale di un processo può o meno corrispondere al modo in cui il processo è stato originariamente definito. I processi tendono a cambiare nel tempo e, indipendentemente da quanto bene siano stati pianificati in origine, possono facilmente deviare. Con il passare del tempo, le deviazioni nel percorso possono diventare la regola, non l'eccezione. La capacità di aggiornare e correggere i processi in modo efficiente dipende dall'ottenere una visibilità completa e in tempo reale su come funzionano i processi.\\
\subsection{Vantaggi del Process Mining}
Ecco alcuni dei principali vantaggi del process mining rispetto ai tradizionali metodi di miglioramento dei processi:
\begin{itemize}
    \item \custombold{Obiettività}: Il process mining offre approfondimenti basati su fatti che provengono da dati reali.
    \item \custombold{Velocità e accuratezza}: Il process mining sostituisce il process mapping, che è manuale, più faticoso e più soggettivo. Il process mining è rapido ed economico, inoltre la sua obiettività ne aumenta l'accuratezza.
    \item \custombold{Evita il rip-and-replace}: Il process mining funziona con sistemi esistenti. Di fatto si tratta di uno "strato" sopra l'infrastruttura IT.
\end{itemize}
\subsection{L'importanza del Process Mining}
I principali motivi per cui il process mining è così importante sono:
\begin{itemize}
    \item \custombold{Visibilità completa sui processi}: Offre una visione oggettiva e in tempo reale basata sui dati IT.
    \item \custombold{Quantificazione dell'impatto}: Con una migliore comprensione delle lacune nei processi, puoi dimostrare il valore prima e dopo aver implementato una soluzione.
    \item \custombold{Coinvolgimento degli stakeholder}: Con suggerimenti di soluzioni basati sui dati e con un ritorno sugli investimenti (ROI) allegato, è molto più facile ottenere consenso e allineamento.
    \item \custombold{Impostazione delle priorità}: Comprendere l'impatto delle lacune specifiche nei processi sui risultati aziendali aiuta a dare priorità alle risorse.
    \item \custombold{Raggiungimento rapido del valore}: Il process mining è facile e veloce da implementare.
\end{itemize}
\subsection{Come funziona il Process Mining}
\subsubsection{Raccolta dei dati}
Al giorno d'oggi, i processi aziendali sono piuttosto complessi. Sono altamente digitalizzati, avvengono all'interno di una vasta gamma di sistemi informativi. I processi potrebbero operare su diversi sistemi che elaborano diversi tipi di dati, gestiti da un grande numero di persone che lavorano in più dipartimenti.\\
Per ottenere una visione dettagliata del tuo processo attraverso il process mining, è necessario che i tuoi strumenti accedano ai dati degli eventi, e ci sono diversi modi per farlo. Un modo è esportare un log degli eventi dal sistema, ottenendo un file di valori separati da virgola che lo strumento di process mining può importare. I metodi di process mining più avanzati utilizzano l'acquisizione dei dati in tempo reale, che sincronizza costantemente i dati del processo più recenti.\\
Nel log degli eventi sono presenti, al minimo, questi tre dati di processo per ciascun evento individuale:
\begin{itemize}
    \item \custombold{Identificatore dell'evento}: Un identificatore univoco che distingue ogni evento specifico.
    \item \custombold{Timestamp}: La data e l'ora esatta in cui l'evento si è verificato.
    \item \custombold{Attività}: La descrizione dell'azione o dell'attività che è stata eseguita durante l'evento.
\end{itemize}
Molti log degli eventi contengono più dettagli rispetto a questi tre elementi. Alcuni dei dettagli aggiuntivi che possono essere inclusi sono:
\begin{itemize}
    \item \custombold{Identificatore del caso}: Un identificatore univoco per ogni processo o caso specifico, che collega una serie di eventi correlati.
    \item \custombold{Utente o risorsa}: Informazioni su chi ha eseguito l'azione o quale risorsa è stata utilizzata.
    \item \custombold{Tipo di evento}: La categoria o il tipo dell'evento, com "inizio", "fine" o altre fasi specifiche del processo.
    \item \custombold{Durata}: Il tempo impiegato per completare l'evento o l'attività.
    \item \custombold{Parametri specifici}: Qualsiasi dato aggiuntivo rilevante per l'evento, come quantità, costi, o altre metriche specifiche del processo.
    \item \custombold{Note o commenti}: Annotazioni aggiuntive che possono fornire ulteriori contesti o dettagli sull'evento.
\end{itemize}
Questi dettagli aggiuntivi aiutano a creare una rappresentazione più completa e accurata del processo, permettendo analisi più approfondite e precise.
\subsubsection{Scoperta dei processi}
La fase nota come "scoperta dei processi" coinvolge l'uso dei log degli eventi per creare una visualizzazione end-to-end del processo. Segue ogni passaggio che ogni caso ha compiuto mentre si muoveva attraverso il ciclo, dall'inizio alla fine. Sovrappone tutti quei percorsi in una sola visualizzazione, una sequenza cronologica degli eventi dal principio alla fine (alcuni chiamano questa "digital twin").\\
Ci sono diversi modi per arrivare da un punto all'altro, con variazioni nel percorso che un processo potrrebbe eseguire. Nel process mining, quei percorsi leggermente diversi sono noti come varianti. Possono esserci centinaia o migliaia di varianti diverse che compaiono sulla mappa dei processi creata dallo strumento di mining.
\subsubsection{Analisi dei processi}
Nel mondo reale, problemi e inefficienze sono comuni, e c'è sempre spazio per migliorare. La fase di analisi dei processi del process mining è dove inizi a scavare nelle cause profonde delle inefficienze dei processi e a quantificare come stanno influenzando gli indicatori chiave di prestazione (KPI).\\
Mentre l'analisi approfondisce le cause delle inefficienze dei processi, ecco alcune delle domande da porsi:
\begin{itemize}
    \item Dove sono i colli di bottiglia nel processo?
    \item Cosa sta causando i ritardi in alcuni casi?
    \item Quali risorse sono sovraccaricate?
    \item Quali attività vengono saltate più spesso?
    \item Quali risorse creano deviazioni?
\end{itemize}
Per quanto riguarda la quantificazione dell'impatto delle varianti e delle inefficienze, queste sono le domande da porsi:
\begin{itemize}
    \item Come impatta questa particolare variante su un certo KPI del processo?
    \item In che modo l'automazione riduce il tempo del ciclo del processo?
    \item Qual è la percentuale di passaggi nel processo che sono automatizzati?
    \item Qual è la percentuale di casi che seguono il processo stabilito e qual è la percentuale che non si conforma?
\end{itemize}
\subsubsection{Benchmarking dei processi}
Mentre si accumulano inforazioni dei processi tramite il process mining, si possono confrontare le prestazioni dei processi su diverse dimensioni. Questo può essere d'aiuto per individuare i migliori esecutori o identificare i punti problematici e, infine, applicare lezioni e best practice da un luogo all'altro, ad esempio tra team, unità aziendali o aree geografiche.
\subsubsection{Verifica della conformità}
La verifica della confermità è la parte del process mining dove si vede la differenza tra il modo ideale nel quale processo dovrebbe essere e il modo nel quale è in realmente in pratica.\\
La verifica della conformità misura dove si trova veramente il processo reale e quale percentuale dei casi conforma effettivamente al processo desiderato. Questo aiuta a determinare quando i passaggi vengono eseguiti nell'ordine sbagliato o forse saltati del tutto. È possibile vedere esattamente quando le cose richiedono più tempo del previsto in determinate fasi del processo.

\subsection{Massimizzare l'utilizzo del Process Mining}
\subsubsection{Trasformare le intuizioni in automazione}
L'automazione può supportare i dipendenti in diversi modi, tra cui:
\begin{itemize}
    \item Ridurre il tempo necessario per completare un processo.
    \item Ridurre gli sprechi.
    \item Migliorare la qualità.
    \item Ridurre i costi.
\end{itemize}
Uno dei probemi principali è identificare quali aree di un processo sono le migliori candidate per l'automazione. L'automazione può ridurre gli sprechi e aumentare l'efficienza, ma non è una garanzia. Se viene automatizzato un processo che alla base è inefficiente, quello che ne risulta è il processo continuerà ad essere rotto, ma più veloce. Nel caso in cui, invece, si codifichino rigidamente le automazioni, c'è la possibilità che si debba ricodificare l'automazione ogni volta che il processo cambia.\\
Questo è un problema comune quando si tratta di alcuni tipi di automazione, come la robotic process automation (RPA). I bot RPA sono eccellenti nell'automazione di compiti semplici di inserimento dati, ma non sono altamente intelligenti o flessibili, quindi quando altre circostanze nel processo cambiano, le automazioni RPA non sempre si adattano. Questo causa malfunzionamenti.\\
La situazione ideale è quando il process mining non solo informa la creazione della tua automazione, ma ne monitora anche l'implementazione per migliorarla continuamente. Ecco dove l'automazione basata sull'intelligenza artificiale e il machine learning può davvero fare la differenza. Questo sistema flessibile è abbastanza adattabile da apprendere e migliorare nel tempo.

\section{Task Mining}
Il task mining è un ambito di studio che si occupa dell'analisi e della comprensione delle attività svolte dagli utenti all'interno di un'organizzazione, con l'obiettivo di ottimizzare e automatizzare i processi aziendali a un livello più granulare rispetto al process mining. Mentre il process mining si concentra sui processi aziendali nel loro complesso, il task mining si concentra sulle singole attività o compiti eseguiti dai dipendenti.\\
Il task mining sfrutta la tecnologia di riconoscimento ottico dei caratteri (OCR). Utilizza l'elaborazione del linguaggio naturale (NLP) ed è supportato da algoritmi di machine learning per comprendere realmente le azioni che le persone compiono sui loro desktop e individua i modelli che influenzano i risultati aziendali.\\
I principi fondamentali di come funziona il task mining:
\begin{itemize}
    \item \custombold{Cattura dati dal desktop}: Si tiene traccia di click, scroll, screenshot, timestamp ed altre azioni.
    \item \custombold{Aggiunta del contensto aziendale}: Si raccoglie, tramite l'OCR, tutto il testo e i numeri sullo schermo per contestualizzare ciò che sta accadendo.
    \item \custombold{Raggruppamento di attività}: Le tecnologie di NLP e intelligenza artificiale comprendono ogni azione e raggruppano le azioni in attività complessive di cui fanno parte.
    \item \custombold{Abbinamento dei dati aziendali}: L'informazione identificativa consente al software di task mining di correlare ciò che l'utente sta facendo con specifici dati aziendali nei sistemi operativi, permettendo di capire realmente come le azioni influiscano sui risultati aziendali.
    \item \custombold{Ottimizzazione del processo}: Tutte le informazioni raccolte possono essere utilizzate per ottimizzare i processi e migliorare le prestazioni aziendali tramite l'uso di un sistema di gestione dell'esecuzione.
\end{itemize}
\subsection{L'importanza del Task Mining}
Il task mining è importante per una serie di motivi che lo rendono benefico per quasi ogni impresa:
\begin{itemize}
    \item Scoprire inefficienze nei modelli di lavoro al di fuori dei sistemi transazionali.
    \item Migliorare la capacità di misurare e ottimizzare la produttività della forza lavoro.
    \item Collegare i processi manuali con i processi aziendali, guidando le prossime migliori azioni sul desktop.
\end{itemize}
Il process mining e il task mining offrono molti vantaggi, in particolare la loro combinazione è molto potente.

\section{Capacità da cercare in uno strumento di Process Mining}
\subsection{Connettere tutti i dati}
È molto importante selezionare uno strumento di process mining che possa raccogliere dati in tempo reale da tutte le fonti necessarie per creare un quadro completo dei tuoi processi. Questo dovrebbe includere sistemi "fatti in casa" e non standard, fogli di calcolo e altri file dove sono archiviati i dati dei processi, insieme a fonti di dati esterne.
\subsubsection{Integrazione con sistemi interni}
\begin{itemize}
    \item Lo strumento dovrebbe essere in grado di connettersi a sistemi ERP, CRM, sistemi di gestione delle risorse umane e qualsiasi altro software aziendale utilizzato internamente.
    \item Dovrebbe supportare API e connettori personalizzati per integrare sistemi non standard o sviluppati internamente.
\end{itemize}
\subsubsection{Supporto per fogli di calcolo e file}
\begin{itemize}
    \item Capacità di importare ed laborare dati da fogli di calcolo.
    \item Supporto per vari formati di file come CSV, JSON, XML.
\end{itemize}
\subsubsection{Integrazione con fonti di dati esterne}
\begin{itemize}
    \item Capacità di raccogliere dati da fonti esterne come servizi cloud, database esterni e API di terze parti.
    \item Supporto per l'integrazione con strumenti di business intelligence e data warehousing per un'analisi più completa.
\end{itemize}
\subsubsection{Acquisizione dei dati in tempo reale}
\begin{itemize}
    \item Capacità di raccogliere dati in tempo reale o quasi reale per garantire che le analisi dei processi siano sempre aggiornate.
    \item utilizzo di tecnologie di streaming dei dati per mantenere sincronizzati i dati tra diverse fonti.
\end{itemize}
\subsubsection{Gestione di dati eterogenei}
\begin{itemize}
    \item Strumenti per normalizzare e unificare i dati provenienti da diverse fonti, garantendo coerenza e qualità dei dati.
    \item Capacità di gestire dati strutturati, semi-strutturati e non strutturati.
\end{itemize}
\subsubsection{Interoperabilità}
\begin{itemize}
    \item Lo strumento dovrebbe essere compatibile con altre piattaforme di analisi e strumenti di data science per consentire un'integrazione fluida.
    \item Supporto per standard di interoperabilità come RESTful APIs e protocolli di trasferimento dati.
\end{itemize}
\subsubsection{Sicurezza e privacy dei dati}
\begin{itemize}
    \item Garanzie che i dati siano raccolti e gestiti in conformità con le normative sulla protezione dei dati.
    \item Funzionalità di crittografia dei dati e controlli di accesso per garantire che solo il personale autorizzato possa accedere ai dati sensibili.
\end{itemize}
\subsubsection{Scalabilità e prestazioni}
\begin{itemize}
    \item Capacità di gestire grandi volumi di dati senza compromettere le prestazioni.
    \item Strumenti per l'ottimizzazione delle prestazioni e la scalabilità orizzontale per supportare la crescita futura.
\end{itemize}
\subsection{Ingestione dei dati}
Una volta identificati i dati per il process mining, è necessario introdurre questi dati nel sistema, il che significa che i prerequisiti critici sono la preparazione, la pulizia e la trasformazione dei dati. Assicurarsi di valutare l'idoneità del modulo di estraione, trasformazione e caricamento (ETL) del potenziale fornitore.
\subsubsection{Preparazione dei dati}
\begin{itemize}
    \item Identificare tutte le fonti di dati rilevanti e raccolta dei dati necessari da ciascuna fonte.
    \item Assicurati che i dati raccolti siano completi e rappresentativi dei processi aziendali che si desidera analizzare.
\end{itemize}
\subsubsection{Pulizia dei dati}
\begin{itemize}
    \item Rimozione dei dati duplicati, corrazione degli errori e gestione dei valori mancanti per garantire che i dati siano accurati e affidabili.
    \item Normalizzare i dati per garantire la coerenza tra diverse fonti di dati.
\end{itemize}
\subsection{Trasformazione dei dati}
\begin{itemize}
    \item Conversione dei dati in un formato appropriato per l'analisi, che potrebbe includere la conversione di tipi di dati, l'unione di dataset e la creazione di nuove variabili derivate.
    \item Aggregazione dei dati ad un livello appropriato per il process mining, mantenendo i dettagli necessari per l'analisi.
\end{itemize}
\subsubsection{Valutazione del modulo ETL del fornitore}
\begin{itemize}
    \item \custombold{Estrazione}: Verificare che il modulo ETL possa estrarre dati da tutte le fonti, inclusi sistemi legacy, database, fogli di calcolo e fonti esterne.
    \item \custombold{Trasformazione}: Assicurarsi che il modulo ETL offra strumenti potenti per la trasformazione dei dati, inclusi strumenti per la pulizia, l'arricchimento e la normalizzazione dei dati.
    \item \custombold{Caricamento}: Verificare che il modulo ETL possa caricare i dati trasformati nel sistema di process mining in modo efficiente, supportando sia carichi batch che carichi in tempo reale.
\end{itemize}
\subsubsection{Automazione del processo ETL}
Considerare l'automazione del processo ETL per ridurre l'intervento manuale e migliorare l'efficienza. Gli strumenti ETL dovrebbero supportare la pianificazione e l'automazione dei job ETL.
\subsubsection{Monitoraggio e gestione della qualità dei dati}
Implementare strumenti di monitoraggio per garantire la qualità dei dati durante tutto il processo ETL. Strumenti di gestione della qualità dei dati possono aiutare a rilevare e correggere problemi di dati in tempo reale.
\subsubsection{Scalabilità e prestazioni del modulo ETL}
\begin{itemize}
    \item Assicurarsi che il modulo ETL sia scalabile e in grado di gestire grandi volumi di dati senza compromettere le prestazioni.
    \item Valutare le capacità di ottimizzazione delle prestazioni del modulo ETL per garantire tempi di caricamento rapidi ed efficienti.
\end{itemize}
\subsubsection{Conformità e sicurezza dei dati}
Verificare che il modulo ETL rispetti le normative sulla protezione dei dati e offra funzionalità di sicurezza robuste, inclusa la crittografia dei dati e il controllo degli accessi.
\subsection{Verifica dei connettori precostruiti}
La maggior parte dei dati utilizzati per il process mining risiede in sistemi standard, come SAP, Oracle e Salesforce. Lo strumento di process mining e gestione dell'esecuzione deve funzionare con tutti i sistemi giusti e deve farlo in modo semplice. Ciò significa disporre di connettori precostruiti che possano caricare i dati rapidamente, dashboard pronte all'uso e una selezione di analisi già create.
\subsubsection{Connettori precostruiti}
\begin{itemize}
    \item Verificare che lo strumento offra connettori precostruiti per i sistemi aziendali più comuni, come SAP, Oracle, Salesforce, Microsoft Dynamics.
    \item Assicurarsi che i connettori supportino l'integrazione senza problemi e siano in grado di gestire dati da diverse versioni e configurazioni dei sistemi.
\end{itemize}
\subsubsection{Caricamento rapido dei dati}
\begin{itemize}
    \item I connettori dovrebbero essere in grado di estrarre e caricare rapidamente i dati dai sistemi di origine, riducendo al minimo il tempo necessario per la preparazione dei dati.
    \item Funzionalità che supportino il caricamento incrementale dei dati per mantenere aggiornati i dataset senza dover eseguire estrazioni complete ogni volta.
\end{itemize}
\subsubsection{Dashboard precostruite}
\begin{itemize}
    \item Lo strumento dovrebbe fornire dashboard precostruite che forniscano visualizzazioni immediate dei dati dei processi.
    \item Le dashboard dovrebbero essere personalizzabili per adattarsi alle specifiche esigenze aziendali e permettere una facile configurazione e modifica.
\end{itemize}
\subsubsection{Analisi pronte all'uso}
\begin{itemize}
    \item Una selezione di analisi già create dovrebbe essere disponibile per aiutare a iniziare rapidamente con il process mining. Queste analisi possono includere il monitoraggio delle prestazioni, l'identificazione dei colli di bottiglia, l'analisi delle varianti.
    \item Le analisi pronte all'uso dovrebbero essere facilmente adattabili per rispondere a domande specifiche o per approfondimenti particolari.
\end{itemize}
\subsubsection{Integrazione multisistema}
\begin{itemize}
    \item Assicurarsi che lo strumento supporti l'integrazione con una vasta gamma di sistemi e fonti di dati. La capacità di lavorare con sistemi diversi è cruciale per ottenere una visione completa dei processi aziendali.
    \item La flessibilità di aggiungere nuovi connettori o integrare sistemi personalizzati è un valore aggiunto.
\end{itemize}
\subsubsection{Facilità d'uso}
Lo strumento dovrebbe essere intuitivo e facile da usare, con una configurazione semplice e documentazione chiara per guidare nell'implementazione.
\subsubsection{Aggiornamenti e manutenzione}
\begin{itemize}
    \item Verificare se il fornitore offre aggiornamenti regolari per i connettori e le funzionalità dello strumento, garantendo che siano sempre compatibili con le ultime versioni dei sistemi di origine.
    \item LA manutenzione dei connettori dovrebbe essere gestita dal fornitore per ridurre il carico di lavoro sul team IT.
\end{itemize}
\subsection{Scoperta e intelligenza dei processi}
Prestare particolare attenzione alle funzionalità di analisi della piattaforma, all'accessibilità per gli utenti aziendali e alla disponibilità di analisi preconfigurate.
\subsubsection{Funzionalità di analisi avanzate}
\begin{itemize}
    \item \custombold{Analisi delle cause principali}: La piattaforma dovrebbe essere in grado di identificare le cause profonde delle inefficienze dei processi, permettendo di affrontare i problemi alla radice.
    \item \custombold{Simulazione dei processi}: La capacità di simulare modifiche ai processi per vedere come influenzerebbero i risultati è cruciale per prendere decisioni informate.
\end{itemize}
\subsubsection{Accessibilità per gli utenti aziendali}
\begin{itemize}
    \item La piattaforma dovrebbe essere intuitiva e facile da utilizzare per gli utenti aziendali, consentendo loro di accedere alle intuizioni senza dover dipendere costantemente dal team IT.
    \item Interfacce utente intuitive e dashboard personalizzabili aiutano a democratizzare l'accesso ai dati e alle analisi.
\end{itemize}
\subsubsection{Analisi preconfigurate}
\begin{itemize}
    \item \custombold{Analisi standard}: La diponibilità di analisi preconfigurate, come monitoraggio delle prestazioni, analisi delle varianti e identificazione dei colli di bottiglia, può accelerare il tempo per ottenere valore.
    \item \custombold{Personalizzazione}: Le analisi preconfigurate dovrebbero essere facilmente adattabili alle specifiche esigenze aziendali.
\end{itemize}
\subsubsection{Personalizzabilità}
\begin{itemize}
    \item La piattaforma dovrebbe consentire la creazione e la personalizzazione di analisi e report per rispondere a domande specifiche o esplorare scenari particolari.
    \item La flessibilità di personalizzare le visualizzazioni e le metriche è essenziale per ottenere intuizioni rilevanti.
\end{itemize}
\subsubsection{Facilità d'uso}
\begin{itemize}
    \item La piattaforma dovrebbe avere una curva di apprendimento minima, con tutorial, documentazione e supporto disponibili per guidare gli utenti attraverso le funzionalità.
    \item Strumenti di drag-and-drop per la creazione di report e dashboard possono rendere l'esperienza utente più fluida ed accessibile.
\end{itemize}
\subsubsection{Supporto per l'intelligenza artificiale(IA) e il machine learning(ML)}
\begin{itemize}
    \item L'integrazione di funzionalità di IA e ML può migliorare l'accuratezza delle analisi efornire previsioni e raccomandazioni basate sui dati.
    \item Strumenti di ML possono aiutare a identificare modelli nascosti e a ottimizzare i processi in modo proattivo.
\end{itemize}
\subsubsection{Integrazione con altri strumenti}
\begin{itemize}
    \item La piattaforma dovrebbe integrarsi facilmente con altri strumenti di business intelligence, data warehousing e strumenti di analisi.
    \item La capacità di esportare dati e risultati in altri formati e sistemi è fondamentale.
\end{itemize}
\subsection{Analisi di processi complessi}
La capacità di visualizzare un processo è il requisito minimo quando si tratta di analisi dei processi. C'è bisogno di strumenti di verifica della conformità e di benchmarking per confrontare le prestazioni del processo rispetto agli standard di riferimento. Alcuni fornitori offrono funzionalità avanzate come la simulazione dei processi e l'analisi incrociata dei processi.
\subsubsection{Verifica della conformità}
\begin{itemize}
    \item \custombold{Confronto con il processo ideale}: La capacità di confrontare i processi attuali con il processo ideale progettato, identificando deviazioni e non conformità.
    \item \custombold{Misurazione della conformità}: Strumenti per misurare il grado di conformità dei processi e identificare le aree in cui sono necessari miglioramenti.
\end{itemize}
\subsubsection{Benchmarking delle prestazioni}
\begin{itemize}
    \item \custombold{Standard di riferimento}: Capacità di confrontare le prestazioni del processo con benchmark interni o di settore.
    \item \custombold{Identificazione delle best practice}: Strumenti per identificare le best practice e applicarle in tutta l'organizzazione.
\end{itemize}
\subsubsection{Simulazione dei processi}
\begin{itemize}
    \item \custombold{Modellazione e simulazione}: Capacità di creare modelli di processo e simulare scenari futuri per valutare l'impatto delle modifiche proposte.
    \item \custombold{Analisi "What-if"}: Strumenti per condurre analisi "What-if" per prevedere i risultati di diverse modifiche ai processi.
\end{itemize}
\subsubsection{Analisi incrociata dei processi}
\begin{itemize}
    \item \custombold{Analisi multi-processo}: Capacità di analizzare più processi contemporaneamente per identificare interazioni e dipendenze.
    \item \custombold{Ottimizzazione globale}: Strumenti per ottimizzare l'intera catena del valore, tenendo conto delle interazioni tra diversi processi.
\end{itemize}
\subsubsection{Visualizzazione avanzata}
\begin{itemize}
    \item \custombold{Mappe di processo interattive}: Visualizzazioni interattive che consentono agli utenti di esplorare i dettagli dei processi e identificare rapidamente problemi e opportunità.
    \item \custombold{Cruscotti personalizzabili}: Dashboard personalizzabili che presentano KPI e metriche chiave in modo chiaro e intuitivo.
\end{itemize}
\subsubsection{Analisi delle cause radice}
\begin{itemize}
    \item \custombold{Identificazione delle cause}: Strumenti avanzati per identificare le cause profonde delle inefficienze e dei problemi nei processi.
    \item \custombold{Raccomandazioni di miglioramento}: Suggerimenti basati sui dati per migliorare le prestazioni dei processi.
\end{itemize}
\subsubsection{Supporto per la collaborazione}
\begin{itemize}
    \item \custombold{Condivisione delle analisi}: Strumenti per condividere facilmente le analisi e le intuizioni con altri membri dell'organizzazione.
    \item \custombold{Feedback e iterazione}: Capacità di raccogliere feedback e iterare rapidamente sui modelli e le simulazioni dei processi.
\end{itemize}
\subsection{Migliorare i processi}
La capacità di eseguire le intuizioni generate dal process mining è probabilmente la caratteristica più importante di tutte.
\subsubsection{Implementazione delle intuizioni}
\begin{itemize}
    \item \custombold{Azioni automatizzate}: La piattaforma dovrebbe consentire di creare e implementare azioni automatizzate basate sulle intuizioni ottenute. Questo può includere l'automazione delle attività ripetitive e la riduzione degli interventi manuali.
    \item \custombold{Flussi di lavoro integrati}: Capacità di integrare flussi di lavoro ottimizzati nei sistemi aziendali esistenti per garantire che le modifiche ai processi siano implementate senza interruzioni.
\end{itemize}
\subsubsection{Monitoraggio e miglioramento continuo}
\begin{itemize}
    \item \custombold{Monitoraggio in tempo reale}: Strumenti per monitorare continuamente le prestazioni dei processi e rilevare eventuali deviazioni o problemi in tempo reale.
    \item \custombold{Feedback loop}: Implementazione di cicli di feedback per raccogliere informazioni sulle modifiche apportate e iterare rapidamente per ulteriori miglioramenti.
\end{itemize}
\subsubsection{Simulazione e pianificazione delle modifiche}
\begin{itemize}
    \item \custombold{Test delle modifiche}:  Capacità di simulare modifiche ai processi prima di implementarle, per valutare l'impatto e prevenire potenziali problemi.
    \item \custombold{Pianificazione delle risorse}: Strumenti per pianificare e allocare le risorse necessarie per implementare efficacemente le modifiche ai processi.
\end{itemize}
\subsubsection{Strumenti di analisi avanzata}
\begin{itemize}
    \item \custombold{Analisi predittiva}: Strumenti che utilizzano analisi predittiva per anticipare problemi e opportunità future, consentendo interventi tempestivi.
    \item \custombold{Analisi delle prestazioni}: Capacità di analizzare le prestazioni dei processi migliorati per misurare l'impatto delle modifiche e dimostrare il valore aggiunto.
\end{itemize}
\subsection{Automatizzare il processo}
Una parte fondamentale del miglioramento è l'automazione. Ogni volta che è possibile, è desiderabile eliminare la necessità di azioni umane e sostituirle con soluzioni e attività automatizzate. Combinando il process mining con l'automazione, è possibile correggere direttamente le inefficienze dei processi.
\subsubsection{Identificazione delle opportunità di automazione}
\begin{itemize}
    \item Utilizzare le intuizioni ottenute dal process mining per identificare le attività ripetitive e dispendiose in termini di tempo che possono essere automatizzate.
    \item Individuare i colli di bottiglia e le inefficienze che possono essere risolti attraverso l'automazione.
\end{itemize}
\subsubsection{Integrazione senza codice}
\begin{itemize}
    \item Scegliere una piattaforma che offra integrazioni senza codice con i tuoi sistemi ERP esistenti, strumenti cloud e software personalizzati.
    \item Utilizzare connettori pre-costruiti per semplificare il processo di integrazione e ridurre il tempo necessario per implementare le soluzioni automatizzate.
\end{itemize}
\subsubsection{Automazione delle attività}
\begin{itemize}
    \item Implementare soluzioni di Robotic Process Automation (RPA) per automatizzare attività ripetitive come l'inserimento dati, l'elaborazione delle transazioni e la gestione delle risposte automatiche.
    \item Utilizzare strumenti di workflow automation per orchestrare sequenze di attività e garantire che le azioni automatizzate siano eseguite nell'ordine corretto.
\end{itemize}
\subsubsection{Monitoraggio e manutenzione}
\begin{itemize}
    \item Implementare strumenti di monitoraggio per assicurarti che le automazioni funzionino correttamente e per rilevare eventuali problemi in tempo reale.
    \item Pianificare la manutenzione regolare delle automazioni per adattarle ai cambiamenti nei processi aziendali e per ottimizzarne le prestazioni.
\end{itemize}
\subsubsection{Analisi e ottimizzazione}
\begin{itemize}
    \item Utilizzare strumenti di analisi per valutare l'impatto delle automazioni sui processi aziendali e per identificare ulteriori opportunità di miglioramento.
    \item Integrare l'intelligenza artificiale e il machine learning per ottimizzare continuamente le automazioni e adattarle alle mutevoli esigenze aziendali.
\end{itemize}
\subsubsection{Formazione e coninvolgimento degli utenti}
\begin{itemize}
    \item Fornire formazione agli utenti aziendali su come utilizzare e interagire con le soluzioni automatizzate.
    \item Coinvolgere gli utenti nel processo di automazione per garantire che le soluzioni implementate soddisfino le loro esigenze e migliorino la loro produttività.
\end{itemize}
\subsubsection{Scalabilità delle soluzioni di automazione}
\begin{itemize}
    \item Assicurarsi che le soluzioni di automazione siano scalabili per supportare la crescita futura dell'azienda.
    \item Pianificare di estendere le automazioni ad altre aree dell'organizzazione per massimizzare i benefici.
\end{itemize}
\subsection{Task mining}
Il desktop process mining (DPM), noto anche come task mining, permette di catturare le attività fuori sistema per migliorare la comprensione dei processi.
\subsubsection{Cattura delle attività desktop}
\begin{itemize}
    \item Utilizzare strumenti di task mining per monitorare e registrare le attività che gli utenti eseguono sul loro desktop, inclusi clic, digitazione, utilizzo di applicazioni e navigazione web.
    \item Strumenti di riconoscimento ottico dei caratteri (OCR) possono essere utilizzati per estrarre testo e dati da screenshot e altre attività visive.
\end{itemize}
\subsubsection{Analisi delle attività}
\begin{itemize}
    \item Le attività registrate vengono analizzate per identificare pattern, inefficienze e colli di bottiglia nelle operazioni quotidiane degli utenti.
    \item L'elaborazione del linguaggio naturale (NLP) e algoritmi di machine learning possono essere utilizzati per comprendere meglio le azioni degli utenti e raggrupparle in attività significative.
\end{itemize}
\subsubsection{Integrazione dei dati}
\begin{itemize}
    \item Combinare i dati di task mining con i dati dei sistemi transazionali per ottenere una visione completa dei processi aziendali, inclusi quelli che avvengono al di fuori dei sistemi IT tradizionali.
    \item L'integrazione dei dati fornisce una mappa dettagliata dei processi end-to-end, includendo tutte le attività manuali e fuori sistema.
\end{itemize}
\subsubsection{Identificazione delle opportunità di automazione}
\begin{itemize}
    \item Utilizzare le intuizioni ottenute dal task mining per identificare attività manuali e ripetitive che possono essere automatizzate.
    \item Implementare soluzioni di automazione come RPA per ridurre il carico di lavoro manuale e migliorare l'efficienza.
\end{itemize}
\subsubsection{Miglioramento della formazione e del supporto}
\begin{itemize}
    \item Le analisi delle attività possono rivelare aree in cui gli utenti necessitano di ulteriore formazione o supporto per migliorare le loro prestazioni.
    \item Utilizzare i dati per sviluppare programmi di formazione mirati e per ottimizzare le risorse di supporto.
\end{itemize}
\subsubsection{Monitoraggio continuo e ottimizzazione}
\begin{itemize}
    \item Implementare un monitoraggio continuo delle attività desktop per rilevare cambiamenti nei comportamenti degli utenti e identificare nuove opportunità di miglioramento.
    \item Adattare e ottimizzare continuamente le strategie di automazione e miglioramento dei processi basandoti sui dati raccolti.
\end{itemize}
\subsubsection{Coninvolgimento degli utenti}
\begin{itemize}
    \item Coinvolgere gli utenti nel processo di task mining per ottenere il loro feedback e garantire che le soluzioni implementate rispondano alle loro esigenze.
    \item Comunicare chiaramente gli obiettivi e i benefici del task mining per ottenere il supporto degli utenti.
\end{itemize}
\subsection{Integrazione con gli strumenti esistenti}
Verificare se il process mining può essere integrato con le tecnologie esistenti come business intellingence (BI), business process management (BPM), integration platform as a service (iPaaS) e robotic process automation (RPA).
\end{document}
