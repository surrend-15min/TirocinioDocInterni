\documentclass{article}
\usepackage[utf8]{inputenc}
\usepackage[includeheadfoot, margin=1em,headheight=2em]{geometry}
\usepackage{titling}
\geometry{a4paper, left=2cm, right=2cm, top=2cm, bottom=2cm}
\usepackage{graphicx}
\providecommand{\versionnumber}{0.0.1}
\usepackage{enumitem}
\usepackage{array}
\usepackage[italian]{babel}
\newcolumntype{P}[1]{>{\centering\arraybackslash}p{#1}}
\renewcommand{\arraystretch}{1.5} % Default value: 1
\setlength{\droptitle}{-6em}

%font
\usepackage[defaultfam,tabular,lining]{montserrat}
\usepackage[T1]{fontenc}
\renewcommand*\oldstylenums[1]{{\fontfamily{Montserrat-TOsF}\selectfont #1}}

%custom bold 
\usepackage[outline]{contour}
\usepackage{xcolor}
\newcommand{\custombold}{\contour{black}}

%table colors
\usepackage{color, colortbl}
\definecolor{Blue}{rgb}{0.51,0.68,0.79}
\definecolor{LightBlue}{rgb}{0.82,0.87,0.90}
\definecolor{LighterBlue}{rgb}{0.93,0.95,0.96}

%Header
\usepackage{fancyhdr, xcolor}
\pagestyle{fancy}
\let\oldheadrule\headrule% Copy \headrule into \oldheadrule
\renewcommand{\headrule}{\color{Blue}\oldheadrule}% Add colour to \headrule
\renewcommand{\headrulewidth}{0.2em}
\fancyhead[L]{Studio delle tecnologie}
\fancyhead[C]{Samuele Vignotto}
\setcounter{secnumdepth}{0}

\title{\Huge{\textbf{Process Mining}}\vspace{-1em}}
\author{Samuele Vignotto}
\date{}
\begin{document}
\maketitle

\newpage
\tableofcontents
\newpage

\section{I Processi}
Un processo è la serie di passaggi che vengono eseguiti in un ordine particolare per raggiungere un risultato specifico.\\
È nel miglior interesse dell'azienda standardizzare i processi più comuni e ad alto volume. In questo modo si garantisce che le cose vengano fatte nel miglior modo possibile, si rende più facile determinare quando le cose vanno storte e si facilita la capacità di formare nuovi o esistenti dipendenti nel modo migliore per svolgere il loro lavoro.\\
Non importa quanto sia buono un processo, ci sono modi per renderlo migliore.\\

\section{Migliorare i processi}
Ecco alcuni dei modi fondamentali per cambiare i processi e renderli migliori:
\begin{itemize}
    \item \custombold{Standardizzazione}: Rendere il processo il più ripetibile possibile e garantire che il processo attuale corrisponda al processo progettato.
    \item \custombold{Snellimento}: Rimuovere dal processo attività rindondanti o non necessarie.
    \item \custombold{Ottimizzazione}: Riprogettare il processo per produrre più valore, come migliorare la qualità o ridurre i costi.
    \item \custombold{Automazione}: Eliminare gli aspetti del processo che richiedono sforzi umani.
\end{itemize}
\end{document}
