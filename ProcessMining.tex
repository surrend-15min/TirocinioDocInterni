\documentclass{article}
\usepackage[utf8]{inputenc}
\usepackage[includeheadfoot, margin=1em,headheight=2em]{geometry}
\usepackage{titling}
\geometry{a4paper, left=2cm, right=2cm, top=2cm, bottom=2cm}
\usepackage{graphicx}
\providecommand{\versionnumber}{1.0.0}
\usepackage{enumitem}
\usepackage{array}
\usepackage[italian]{babel}
\newcolumntype{P}[1]{>{\centering\arraybackslash}p{#1}}
\renewcommand{\arraystretch}{1.5} % Default value: 1
\setlength{\droptitle}{-6em}

%font
\usepackage[defaultfam,tabular,lining]{montserrat}
\usepackage[T1]{fontenc}
\renewcommand*\oldstylenums[1]{{\fontfamily{Montserrat-TOsF}\selectfont #1}}

%custom bold 
\usepackage[outline]{contour}
\usepackage{xcolor}
\newcommand{\custombold}{\contour{black}}

%table colors
\usepackage{color, colortbl}
\definecolor{Blue}{rgb}{0.51,0.68,0.79}
\definecolor{LightBlue}{rgb}{0.82,0.87,0.90}
\definecolor{LighterBlue}{rgb}{0.93,0.95,0.96}

%Header
\usepackage{fancyhdr, xcolor}
\pagestyle{fancy}
\let\oldheadrule\headrule% Copy \headrule into \oldheadrule
\renewcommand{\headrule}{\color{Blue}\oldheadrule}% Add colour to \headrule
\renewcommand{\headrulewidth}{0.2em}
\fancyhead[L]{Studio Process Mining}
\fancyhead[C]{Samuele Vignotto}
\fancyhead[R]{\includegraphics[height=1cm]{Logo/Y_LOGO-SOLO.png}}
\setcounter{secnumdepth}{0}

\title{\Huge{\textbf{Process Mining}}\vspace{-1em}}
\author{Samuele Vignotto}
\date{}
\begin{document}
\maketitle
\begin{figure}[h]
  \centering
  \includegraphics[width=6cm, height=6cm]{Logo/Y_LOGO-SOLO.png}
  \label{fig:immagine}
\end{figure}

\newpage
\tableofcontents
\newpage

\section{I Processi}
Un processo è la serie di passaggi che vengono eseguiti in un ordine particolare per raggiungere un risultato specifico.\\
È nel miglior interesse dell'azienda standardizzare i processi più comuni e ad alto volume. In questo modo si garantisce che le cose vengano fatte nel miglior modo possibile, si rende più facile determinare quando le cose vanno storte e si facilita la capacità di formare nuovi o esistenti dipendenti nel modo migliore per svolgere il loro lavoro.\\
Non importa quanto sia buono un processo, ci sono modi per renderlo migliore.\\

\subsection{Migliorare i processi}
Ecco alcuni dei modi fondamentali per cambiare i processi e renderli migliori:
\begin{itemize}
    \item \custombold{Standardizzazione}: rendere il processo il più ripetibile possibile e garantire che il processo attuale corrisponda al processo progettato;
    \item \custombold{Snellimento}: rimuovere dal processo attività rindondanti o non necessarie;
    \item \custombold{Ottimizzazione}: riprogettare il processo per produrre più valore, come migliorare la qualità o ridurre i costi;
    \item \custombold{Automazione}: eliminare gli aspetti del processo che richiedono sforzi umani.
\end{itemize}

\section{Process Mining}
Il process mining è un ambito di studio della gestione dei processi aziendali che si occupa dell'analisi dei processi operativi di un'organizzazione basandosi su dati generati dai sistemi informatici. Il suo obiettivo principale è scoprire, monitorare e migliorare i processi reali attraverso l'estrazione e l'analisi dei log di eventi registrati dai sistemi informativi.\\
Ciò che rende possibile il process mining è il fatto che molti dei processi aziendali più diffusi vengono eseguiti tramite sistemi informativi. In senso generale, l'estrazione comporta la ricerca e l'estrazione di qualcosa di valore. Il process mining estrae conoscenze preziose dai log degli eventi prodotti dai sistemi informativi.\\
Per comprendere al meglio il funzionamento del process mining si deve ricordare che un processo è una serie di azioni o passaggi che vanno da un punto di partenza a un punto di arrivo riconosciuto. Questi passaggi possono essere ripetuti e possono essere migliorati nella speranza di passare dal punto di partenza al punto di arrivo nel modo più efficiente e coerente possibile. Quando questi passaggi avvengono in un sistema transazionale, lasciano un'impronta digitale sotto forma di dati dei log degli eventi. Il process mining estrae quei dati e li utilizza per creare un'immagine di come i processi appaiono effettivamente nella pratica.\\
L'aspetto reale di un processo può o meno corrispondere al modo in cui il processo è stato originariamente definito. I processi tendono a cambiare nel tempo e, indipendentemente da quanto bene siano stati pianificati in origine, possono facilmente deviare. Con il passare del tempo, le deviazioni nel percorso possono diventare la regola, non l'eccezione. La capacità di aggiornare e correggere i processi in modo efficiente dipende dall'ottenere una visibilità completa e in tempo reale su come funzionano i processi.\\
\subsection{Vantaggi del Process Mining}
Ecco alcuni dei principali vantaggi del process mining rispetto ai tradizionali metodi di miglioramento dei processi:
\begin{itemize}
    \item \custombold{Obiettività}: il process mining offre approfondimenti basati su fatti che provengono da dati reali;
    \item \custombold{Velocità e accuratezza}: il process mining sostituisce il process mapping, che è manuale, più faticoso e più soggettivo. Il process mining è rapido ed economico, inoltre la sua obiettività ne aumenta l'accuratezza;
    \item \custombold{Evita il rip-and-replace}: il process mining funziona con sistemi esistenti. Di fatto si tratta di uno "strato" sopra l'infrastruttura IT.
\end{itemize}
\subsection{L'importanza del Process Mining}
I principali motivi per cui il process mining è così importante sono:
\begin{itemize}
    \item \custombold{Visibilità completa sui processi}: offre una visione oggettiva e in tempo reale basata sui dati IT.
    \item \custombold{Quantificazione dell'impatto}: con una migliore comprensione delle lacune nei processi, puoi dimostrare il valore prima e dopo aver implementato una soluzione;
    \item \custombold{Coinvolgimento degli stakeholder}: con suggerimenti di soluzioni basati sui dati e con un ritorno sugli investimenti (ROI) allegato, è molto più facile ottenere consenso e allineamento;
    \item \custombold{Impostazione delle priorità}: comprendere l'impatto delle lacune specifiche nei processi sui risultati aziendali aiuta a dare priorità alle risorse;
    \item \custombold{Raggiungimento rapido del valore}: il process mining è facile e veloce da implementare.
\end{itemize}
\subsection{Come funziona il Process Mining}
\subsubsection{Raccolta dei dati}
Al giorno d'oggi, i processi aziendali sono piuttosto complessi. Sono altamente digitalizzati, avvengono all'interno di una vasta gamma di sistemi informativi. I processi potrebbero operare su diversi sistemi che elaborano diversi tipi di dati, gestiti da un grande numero di persone che lavorano in più dipartimenti.\\
Per ottenere una visione dettagliata del tuo processo attraverso il process mining, è necessario che i tuoi strumenti accedano ai dati degli eventi, e ci sono diversi modi per farlo. Un modo è esportare un log degli eventi dal sistema, ottenendo un file di valori separati da virgola che lo strumento di process mining può importare. I metodi di process mining più avanzati utilizzano l'acquisizione dei dati in tempo reale, che sincronizza costantemente i dati del processo più recenti.\\
Nel log degli eventi sono presenti, al minimo, questi tre dati di processo per ciascun evento individuale:
\begin{itemize}
    \item \custombold{Identificatore dell'evento}: un identificatore univoco che distingue ogni evento specifico;
    \item \custombold{Timestamp}: la data e l'ora esatta in cui l'evento si è verificato;
    \item \custombold{Attività}: la descrizione dell'azione o dell'attività che è stata eseguita durante l'evento.
\end{itemize}
Molti log degli eventi contengono più dettagli rispetto a questi tre elementi. Alcuni dei dettagli aggiuntivi che possono essere inclusi sono:
\begin{itemize}
    \item \custombold{Identificatore del caso}: un identificatore univoco per ogni processo o caso specifico, che collega una serie di eventi correlati;
    \item \custombold{Utente o risorsa}: informazioni su chi ha eseguito l'azione o quale risorsa è stata utilizzata;
    \item \custombold{Tipo di evento}: la categoria o il tipo dell'evento, com "inizio", "fine" o altre fasi specifiche del processo;
    \item \custombold{Durata}: il tempo impiegato per completare l'evento o l'attività;
    \item \custombold{Parametri specifici}: qualsiasi dato aggiuntivo rilevante per l'evento, come quantità, costi, o altre metriche specifiche del processo;
    \item \custombold{Note o commenti}: annotazioni aggiuntive che possono fornire ulteriori contesti o dettagli sull'evento.
\end{itemize}
Questi dettagli aggiuntivi aiutano a creare una rappresentazione più completa e accurata del processo, permettendo analisi più approfondite e precise.
\subsubsection{Scoperta dei processi}
La fase nota come "scoperta dei processi" coinvolge l'uso dei log degli eventi per creare una visualizzazione end-to-end del processo. Segue ogni passaggio che ogni caso ha compiuto mentre si muoveva attraverso il ciclo, dall'inizio alla fine. Sovrappone tutti quei percorsi in una sola visualizzazione, una sequenza cronologica degli eventi dal principio alla fine (alcuni chiamano questa "digital twin").\\
Ci sono diversi modi per arrivare da un punto all'altro, con variazioni nel percorso che un processo potrrebbe eseguire. Nel process mining, quei percorsi leggermente diversi sono noti come varianti. Possono esserci centinaia o migliaia di varianti diverse che compaiono sulla mappa dei processi creata dallo strumento di mining.
\subsubsection{Analisi dei processi}
Nel mondo reale, problemi e inefficienze sono comuni, e c'è sempre spazio per migliorare. La fase di analisi dei processi nel process mining permette di esaminare a fondo le cause profonde delle inefficienze e di quantificare il loro impatto sugli indicatori chiave di prestazione (KPI). Durante questa fase, è fondamentale identificare i colli di bottiglia che rallentano il processo, comprendere le cause dei ritardi in determinati casi e riconoscere quali risorse sono sovraccaricate. Inoltre, è importante determinare quali attività vengono saltate più frequentemente e quali risorse causano deviazioni rispetto al processo pianificato.\\
Quantificare l'impatto delle varianti e delle inefficienze è altrettanto cruciale. Si deve valutare come una particolare variante influisce su specifici KPI del processo e analizzare in che modo l'automazione contribuisce a ridurre il tempo del ciclo. È essenziale anche comprendere la percentuale di passaggi nel processo che sono automatizzati e la proporzione di casi che seguono il processo stabilito rispetto a quelli che non si conformano. Questo tipo di analisi approfondita consente di identificare aree di miglioramento e implementare soluzioni efficaci per ottimizzare i processi aziendali.

\subsubsection{Benchmarking dei processi}
Mentre si accumulano inforazioni dei processi tramite il process mining, si possono confrontare le prestazioni dei processi su diverse dimensioni. Questo può essere d'aiuto per individuare i migliori esecutori o identificare i punti problematici e, infine, applicare lezioni e best practice da un luogo all'altro, ad esempio tra team, unità aziendali o aree geografiche.
\subsubsection{Verifica della conformità}
La verifica della confermità è la parte del process mining dove si vede la differenza tra il modo ideale nel quale processo dovrebbe essere e il modo nel quale è in realmente in pratica.\\
La verifica della conformità misura dove si trova veramente il processo reale e quale percentuale dei casi conforma effettivamente al processo desiderato. Questo aiuta a determinare quando i passaggi vengono eseguiti nell'ordine sbagliato o forse saltati del tutto. È possibile vedere esattamente quando le cose richiedono più tempo del previsto in determinate fasi del processo.

\subsection{Massimizzare l'utilizzo del Process Mining}
\subsubsection{Trasformare le intuizioni in automazione}
L'automazione può supportare i dipendenti in diversi modi, tra cui:
\begin{itemize}
    \item Ridurre il tempo necessario per completare un processo.
    \item Ridurre gli sprechi.
    \item Migliorare la qualità.
    \item Ridurre i costi.
\end{itemize}
Uno dei probemi principali è identificare quali aree di un processo sono le migliori candidate per l'automazione. L'automazione può ridurre gli sprechi e aumentare l'efficienza, ma non è una garanzia. Se viene automatizzato un processo che alla base è inefficiente, quello che ne risulta è il processo continuerà ad essere rotto, ma più veloce. Nel caso in cui, invece, si codifichino rigidamente le automazioni, c'è la possibilità che si debba ricodificare l'automazione ogni volta che il processo cambia.\\
Questo è un problema comune quando si tratta di alcuni tipi di automazione, come la robotic process automation (RPA). I bot RPA sono eccellenti nell'automazione di compiti semplici di inserimento dati, ma non sono altamente intelligenti o flessibili, quindi quando altre circostanze nel processo cambiano, le automazioni RPA non sempre si adattano. Questo causa malfunzionamenti.\\
La situazione ideale è quando il process mining non solo informa la creazione della tua automazione, ma ne monitora anche l'implementazione per migliorarla continuamente. Ecco dove l'automazione basata sull'intelligenza artificiale e il machine learning può davvero fare la differenza. Questo sistema flessibile è abbastanza adattabile da apprendere e migliorare nel tempo.

\section{Task Mining}
Il task mining è un ambito di studio che si occupa dell'analisi e della comprensione delle attività svolte dagli utenti all'interno di un'organizzazione, con l'obiettivo di ottimizzare e automatizzare i processi aziendali a un livello più granulare rispetto al process mining. Mentre il process mining si concentra sui processi aziendali nel loro complesso, il task mining si concentra sulle singole attività o compiti eseguiti dai dipendenti.\\
Il task mining sfrutta la tecnologia di riconoscimento ottico dei caratteri (OCR). Utilizza l'elaborazione del linguaggio naturale (NLP) ed è supportato da algoritmi di machine learning per comprendere realmente le azioni che le persone compiono sui loro desktop e individua i modelli che influenzano i risultati aziendali.\\
I principi fondamentali di come funziona il task mining:
\begin{itemize}
    \item \custombold{Cattura dati dal desktop}: Si tiene traccia di click, scroll, screenshot, timestamp ed altre azioni.
    \item \custombold{Aggiunta del contensto aziendale}: Si raccoglie, tramite l'OCR, tutto il testo e i numeri sullo schermo per contestualizzare ciò che sta accadendo.
    \item \custombold{Raggruppamento di attività}: Le tecnologie di NLP e intelligenza artificiale comprendono ogni azione e raggruppano le azioni in attività complessive di cui fanno parte.
    \item \custombold{Abbinamento dei dati aziendali}: L'informazione identificativa consente al software di task mining di correlare ciò che l'utente sta facendo con specifici dati aziendali nei sistemi operativi, permettendo di capire realmente come le azioni influiscano sui risultati aziendali.
    \item \custombold{Ottimizzazione del processo}: Tutte le informazioni raccolte possono essere utilizzate per ottimizzare i processi e migliorare le prestazioni aziendali tramite l'uso di un sistema di gestione dell'esecuzione.
\end{itemize}
\subsection{L'importanza del Task Mining}
Il task mining è importante per una serie di motivi che lo rendono benefico per quasi ogni impresa:
\begin{itemize}
    \item Scoprire inefficienze nei modelli di lavoro al di fuori dei sistemi transazionali.
    \item Migliorare la capacità di misurare e ottimizzare la produttività della forza lavoro.
    \item Collegare i processi manuali con i processi aziendali, guidando le prossime migliori azioni sul desktop.
\end{itemize}
Il process mining e il task mining offrono molti vantaggi, in particolare la loro combinazione è molto potente.

\section{Capacità da cercare in uno strumento di Process Mining}
\subsection{Connettere tutti i dati}
È essenziale scegliere uno strumento di process mining che possa raccogliere dati in tempo reale da tutte le fonti necessarie per avere una visione completa dei processi aziendali. Questo include sistemi personalizzati, fogli di calcolo, file vari e fonti di dati esterne. Lo strumento deve essere in grado di connettersi a sistemi ERP, CRM, software di gestione delle risorse umane e qualsiasi altro software aziendale utilizzato. Deve supportare API e connettori personalizzati per integrare sistemi non standard o sviluppati internamente, con la capacità di importare ed elaborare dati da fogli di calcolo in vari formati come CSV, JSON e XML.\\
É fondamentale che lo strumento possa raccogliere dati da fonti esterne come servizi cloud, database esterni e API di terze parti, supportando l'integrazione con strumenti di business intelligence e data warehousing per un'analisi più completa. Per garantire che le analisi dei processi siano sempre aggiornate, lo strumento deve raccogliere dati in tempo reale o quasi reale, utilizzando tecnologie di streaming dei dati per mantenere sincronizzati i dati tra diverse fonti.\\
Lo strumento deve poter normalizzare e unificare i dati provenienti da diverse fonti, garantendo coerenza e qualità, e gestire dati strutturati, semi-strutturati e non strutturati. Lo strumento dovrebbe essere compatibile con altre piattaforme di analisi e strumenti di data science per garantire un'integrazione fluida, supportando standard di interoperabilità come RESTful API e protocolli di trasferimento dati. La sicurezza e la privacy dei dati sono cruciali: i dati devono essere raccolti e gestiti in conformità con le normative sulla protezione dei dati, con funzionalità di crittografia e controlli di accesso per garantire che solo il personale autorizzato possa accedere ai dati sensibili. Infine, lo strumento deve essere scalabile, capace di gestire grandi volumi di dati senza compromettere le prestazioni, e deve offrire strumenti per l'ottimizzazione delle prestazioni e la scalabilità orizzontale per supportare la crescita futura.

\subsection{Ingestione dei dati}
Dopo aver identificato i dati necessari per il process mining, è cruciale introdurli nel sistema. Questo richiede la preparazione, la pulizia e la trasformazione dei dati, assicurandosi che siano completi e rappresentativi dei processi aziendali. La preparazione dei dati comporta l'identificazione di tutte le fonti rilevanti e la raccolta dei dati necessari da ciascuna di esse. La pulizia dei dati prevede la rimozione di duplicati, la correzione degli errori e la gestione dei valori mancanti per garantire l'accuratezza e l'affidabilità dei dati, normalizzandoli per mantenere la coerenza tra diverse fonti.\\
I dati devono essere convertiti in un formato appropriato per l'analisi, il che potrebbe includere la conversione di tipi di dati, l'unione di dataset e la creazione di nuove variabili derivate. È essenziale valutare l'idoneità del modulo di estrazione, trasformazione e caricamento (ETL) del fornitore. Il modulo ETL deve poter estrarre dati da tutte le fonti, inclusi sistemi legacy, database, fogli di calcolo e fonti esterne. Deve offrire strumenti potenti per la trasformazione dei dati, inclusi strumenti per la pulizia, l'arricchimento e la normalizzazione, e deve poter caricare i dati trasformati nel sistema di process mining in modo efficiente, supportando sia carichi batch che in tempo reale.\\
Considerare l'automazione del processo ETL può ridurre l'intervento manuale e migliorare l'efficienza. Gli strumenti ETL dovrebbero supportare la pianificazione e l'automazione dei job ETL. È importante implementare strumenti di monitoraggio per garantire la qualità dei dati durante tutto il processo ETL, rilevando e correggendo problemi di dati in tempo reale. Il modulo ETL deve essere scalabile, in grado di gestire grandi volumi di dati senza compromettere le prestazioni, e offrire capacità di ottimizzazione delle prestazioni per garantire tempi di caricamento rapidi ed efficienti. Infine, deve rispettare le normative sulla protezione dei dati e offrire funzionalità di sicurezza robuste, inclusa la crittografia e il controllo degli accessi.

\subsection{Verifica dei connettori precostruiti}
La maggior parte dei dati per il process mining risiede in sistemi standard come SAP, Oracle e Salesforce. Lo strumento deve funzionare con tutti questi sistemi in modo semplice, disponendo di connettori precostruiti che possano caricare i dati rapidamente, dashboard pronte all'uso e una selezione di analisi già create. È essenziale verificare che lo strumento offra connettori precostruiti per i sistemi aziendali più comuni, come SAP, Oracle, Salesforce e Microsoft Dynamics, e che questi connettori supportino l'integrazione senza problemi, gestendo dati da diverse versioni e configurazioni.\\
I connettori devono essere in grado di estrarre e caricare rapidamente i dati dai sistemi di origine, riducendo al minimo il tempo necessario per la preparazione dei dati. Deve essere disponibile il supporto per il caricamento incrementale dei dati per mantenere aggiornati i dataset senza dover eseguire sempre estrazioni complete. Lo strumento dovrebbe fornire dashboard precostruite che offrano visualizzazioni immediate dei dati dei processi, personalizzabili per adattarsi alle specifiche esigenze aziendali e permettere una facile configurazione.\\
Una selezione di analisi già create dovrebbe essere disponibile per iniziare rapidamente con il process mining. Le analisi preconfigurate devono essere facilmente adattabili alle specifiche esigenze aziendali. Queste analisi possono includere il monitoraggio delle prestazioni, l'identificazione dei colli di bottiglia e l'analisi delle varianti. Le analisi pronte all'uso dovrebbero essere facilmente adattabili per rispondere a domande specifiche o per approfondimenti particolari. È importante che lo strumento supporti l'integrazione con una vasta gamma di sistemi e fonti di dati, lavorando con sistemi diversi per ottenere una visione completa dei processi aziendali. Un valore aggiunto è la flessibilità di aggiungere nuovi connettori o integrare sistemi personalizzati.\\
Lo strumento deve essere intuitivo e facile da usare, con una configurazione semplice e documentazione chiara per guidare l'implementazione. Verificare se il fornitore offre aggiornamenti regolari per i connettori e le funzionalità dello strumento è fondamentale, così da garantire che siano sempre compatibili con le ultime versioni dei sistemi di origine. La manutenzione dei connettori dovrebbe essere gestita dal fornitore per ridurre il carico di lavoro sul team IT.

\subsection{Efficienza e Accessibilità}
La piattaforma deve essere in grado di identificare le cause profonde delle inefficienze dei processi, permettendo di affrontare i problemi alla radice. La capacità di simulare modifiche ai processi per vedere come influenzerebbero i risultati è cruciale per prendere decisioni informate. 
La piattaforma deve essere intuitiva e facile da utilizzare per gli utenti aziendali, consentendo loro di accedere alle intuizioni senza dover dipendere costantemente dal team IT. Interfacce utente intuitive e dashboard personalizzabili aiutano a democratizzare l'accesso ai dati e alle analisi. La disponibilità di analisi preconfigurate, come il monitoraggio delle prestazioni, l'analisi delle varianti e l'identificazione dei colli di bottiglia, può accelerare il tempo per ottenere valore. 
La piattaforma deve consentire la creazione e la personalizzazione delle analisi e dei report per rispondere a domande specifiche o esplorare scenari particolari. La flessibilità di personalizzare le visualizzazioni e le metriche è essenziale per ottenere intuizioni rilevanti. La piattaforma deve avere una curva di apprendimento minima, con tutorial, documentazione e supporto disponibili per guidare gli utenti attraverso le funzionalità. Strumenti di drag-and-drop per la creazione di report e dashboard possono rendere l'esperienza utente più fluida e accessibile.\\
L'integrazione di funzionalità di intelligenza artificiale (IA) e machine learning (ML) può migliorare l'accuratezza delle analisi e fornire previsioni e raccomandazioni basate sui dati. Strumenti di ML possono aiutare a identificare modelli nascosti e a ottimizzare i processi in modo proattivo.

\subsection{Analisi di processi complessi}
Oltre alla visualizzazione dei processi, è necessario disporre di strumenti per la verifica della conformità e il benchmarking per confrontare le prestazioni del processo rispetto agli standard di riferimento. Alcuni fornitori offrono funzionalità avanzate come la simulazione dei processi e l'analisi incrociata dei processi. La capacità di confrontare i processi attuali con il processo ideale progettato, identificando deviazioni e non conformità, è fondamentale. Strumenti per misurare il grado di conformità dei processi e identificare le aree in cui sono necessari miglioramenti sono indispensabili.\\
La capacità di confrontare le prestazioni del processo con benchmark interni o di settore e identificare le best practice è essenziale per migliorare i processi aziendali. La possibilità di creare modelli di processo e simulare scenari futuri per valutare l'impatto delle modifiche proposte, nonché condurre analisi "What-if" per prevedere i risultati di diverse modifiche ai processi, è di grande valore.\\
L'analisi multi-processo consente di analizzare più processi contemporaneamente per identificare interazioni e dipendenze. Visualizzazioni interattive che permettono di esplorare i dettagli dei processi e identificare rapidamente problemi e opportunità, insieme a dashboard personalizzabili che presentano KPI e metriche chiave in modo chiaro e intuitivo, sono strumenti potenti per l'analisi avanzata dei processi.\\
La piattaforma deve mettere a disposizione la possibilità di identificare le cause profonde delle inefficienze e dei problemi nei processi, fornendo raccomandazioni basate sui dati per migliorare le prestazioni, sono essenziali. Inoltre, la possibilità di condividere facilmente le analisi e le intuizioni con altri membri dell'organizzazione e di raccogliere feedback per iterare rapidamente sui modelli e le simulazioni dei processi favorisce un ambiente di collaborazione e miglioramento continuo.

\subsection{Migliorare i processi}
La capacità di eseguire le intuizioni generate dal process mining è probabilmente la caratteristica più importante di tutte. La piattaforma deve consentire di creare e implementare azioni automatizzate basate sulle intuizioni ottenute, come l'automazione delle attività ripetitive e la riduzione degli interventi manuali. È fondamentale poter integrare flussi di lavoro ottimizzati nei sistemi aziendali esistenti per garantire che le modifiche ai processi siano implementate senza interruzioni.\\
Strumenti per monitorare continuamente le prestazioni dei processi e rilevare eventuali deviazioni o problemi in tempo reale sono indispensabili. Implementare cicli di feedback per raccogliere informazioni sulle modifiche apportate e iterare rapidamente per ulteriori miglioramenti è cruciale per un miglioramento continuo.\\
La capacità di simulare modifiche ai processi prima di implementarle, per valutare l'impatto e prevenire potenziali problemi, insieme a strumenti per pianificare e allocare le risorse necessarie per implementare efficacemente le modifiche, è essenziale. Strumenti di analisi predittiva che utilizzano i dati per anticipare problemi e opportunità future, consentendo interventi tempestivi, e la capacità di analizzare le prestazioni dei processi migliorati per misurare l'impatto delle modifiche e dimostrare il valore aggiunto, sono fondamentali per il miglioramento dei processi.

\subsection{Automatizzare il processo}
L'automazione è una parte fondamentale del miglioramento dei processi. Ogni volta che è possibile, è preferibile eliminare le azioni umane e sostituirle con soluzioni automatizzate. Utilizzare le intuizioni ottenute dal process mining per identificare le attività ripetitive e dispendiose in termini di tempo che possono essere automatizzate, individuando i colli di bottiglia e le inefficienze che possono essere risolti attraverso l'automazione.\\
La piattaforma deve offrire connettori pre-costruiti che semplifichino il processo di integrazione e riducano il tempo necessario per implementare le soluzioni automatizzate.La piattaforma deve disporre di strumenti di workflow automation per orchestrare sequenze di attività e garantire che le azioni automatizzate siano eseguite nell'ordine corretto.\\
Implementare strumenti di monitoraggio per assicurarsi che le automazioni funzionino correttamente e per rilevare eventuali problemi in tempo reale. Pianificare la manutenzione regolare delle automazioni per adattarle ai cambiamenti nei processi aziendali e per ottimizzarne le prestazioni. Utilizzare strumenti di analisi per la valutazio l'impatto delle automazioni sui processi aziendali e per identificare ulteriori opportunità di miglioramento. Integrazione dell'IA e del ML per ottimizzare continuamente le automazioni e adattarle alle mutevoli esigenze aziendali.\\
Implementare le automazioni comporta di fornire formazione agli utenti aziendali su come utilizzare e interagire con le soluzioni automatizzate, coinvolgendoli nel processo di automazione per garantire che le soluzioni implementate soddisfino le loro esigenze e migliorino la loro produttività. É necessario assicurarsi che le soluzioni di automazione siano scalabili per supportare la crescita futura dell'azienda, pianificando di estendere le automazioni ad altre aree dell'organizzazione per massimizzare i benefici.

\subsection{Task mining}
La piattaforma deve consentire la possibilità di poter analizzare le attività registrate tramite il task mining per identificare pattern, inefficienze e colli di bottiglia nelle operazioni quotidiane degli utenti. L'elaborazione del linguaggio naturale e algoritmi di machine learning possono essere utilizzati per comprendere meglio le azioni degli utenti e raggrupparle in attività significative. Combinare i dati di task mining con i dati dei sistemi transazionali permette di ottenere una visione completa dei processi aziendali, inclusi quelli che avvengono al di fuori dei sistemi IT tradizionali. L'integrazione dei dati fornisce una mappa dettagliata dei processi end-to-end, includendo tutte le attività manuali e fuori sistema. Le analisi delle attività possono rivelare aree in cui gli utenti necessitano di ulteriore formazione o supporto per migliorare le loro prestazioni. Utilizzare i dati per sviluppare programmi di formazione mirati e per ottimizzare le risorse di supporto.\\
Implementare un monitoraggio continuo delle attività desktop per rilevare cambiamenti nei comportamenti degli utenti e identificare nuove opportunità di miglioramento. Adattare e ottimizzare continuamente le strategie di automazione e miglioramento dei processi basandosi sui dati raccolti. Coinvolgere gli utenti nel processo di task mining per ottenere il loro feedback e garantire che le soluzioni implementate rispondano alle loro esigenze. Comunicare chiaramente gli obiettivi e i benefici del task mining per ottenere il supporto degli utenti.

\subsection{Integrazione con gli strumenti esistenti}
Infine, è essenziale verificare se il process mining può essere integrato con le tecnologie esistenti come business intelligence, business process management, integration platform as a service e robotic process automation.

\end{document}
