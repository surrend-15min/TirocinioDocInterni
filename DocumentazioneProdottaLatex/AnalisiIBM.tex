\documentclass{article}
\usepackage[utf8]{inputenc}
\usepackage[includeheadfoot, margin=1em,headheight=2em]{geometry}
\usepackage{titling}
\geometry{a4paper, left=2cm, right=2cm, top=2cm, bottom=2cm}
\usepackage{graphicx}
\usepackage{float}
\providecommand{\versionnumber}{1.0.0}
\usepackage{enumitem}
\usepackage{array}
\usepackage[italian]{babel}
\newcolumntype{P}[1]{>{\centering\arraybackslash}p{#1}}
\renewcommand{\arraystretch}{1.5} % Default value: 1
\setlength{\droptitle}{-6em}

%font
\usepackage[defaultfam,tabular,lining]{montserrat}
\usepackage[T1]{fontenc}
\renewcommand*\oldstylenums[1]{{\fontfamily{Montserrat-TOsF}\selectfont #1}}

%custom bold 
\usepackage[outline]{contour}
\usepackage{xcolor}
\newcommand{\custombold}{\contour{black}}

%table colors
\usepackage{color, colortbl}
\definecolor{Blue}{rgb}{0.51,0.68,0.79}
\definecolor{LightBlue}{rgb}{0.82,0.87,0.90}
\definecolor{LighterBlue}{rgb}{0.93,0.95,0.96}

%Header
\usepackage{fancyhdr, xcolor}
\pagestyle{fancy}
\let\oldheadrule\headrule% Copy \headrule into \oldheadrule
\renewcommand{\headrule}{\color{Blue}\oldheadrule}% Add colour to \headrule
\renewcommand{\headrulewidth}{0.2em}
\fancyhead[L]{Studio IBM Process Mining}
\fancyhead[C]{Samuele Vignotto}
\fancyhead[R]{\includegraphics[height=1cm]{Logo/Y_LOGO-SOLO.png}}
\setcounter{secnumdepth}{0}

\title{\Huge{\textbf{IBM Process Mining}}\vspace{-1em}}
\author{Samuele Vignotto}
\date{}
\begin{document}
\maketitle
\begin{figure}[h]
  \centering
  \includegraphics[width=6cm, height=6cm]{Logo/Y_LOGO-SOLO.png}
  \label{fig:immagine}
\end{figure}

\newpage
\tableofcontents
\newpage

\section{Introduzione}

IBM Process Mining è uno strumento per l'analisi e l'ottimizzazione dei processi aziendali, progettato per fornire una comprensione di come i processi vengono realmente eseguiti all'interno di un'organizzazione. Grazie alla capacità di sfruttare i dati storici raccolti da diverse fonti aziendali, IBM Process Mining permette di ricostruire il flusso di lavoro effettivo, individuando inefficienze e deviazioni rispetto ai processi ideali.\\
La visualizzazione dettagliata dei processi è uno degli aspetti chiave della soluzione, consentendo agli utenti di vedere come le attività si susseguono nel tempo. Inoltre, lo strumento identifica automaticamente i colli di bottiglia e le ridondanze.\\
Un'altra funzione essenziale di IBM Process Mining è la sua integrazione con tecnologie di automazione e intelligenza artificiale. Questo permette di creare flussi di lavoro automatizzati che semplificano le attività ripetitive, riducendo l'intervento manuale e minimizzando il rischio di errori. Inoltre, lo strumento supporta un monitoraggio continuo delle prestazioni dei processi, consentendo una gestione proattiva e miglioramenti progressivi in tempo reale.\\
Le funzionalità di simulazione offerte da IBM Process Mining permettono di testare vari scenari e prevedere l’impatto di eventuali modifiche, facilitando così un processo decisionale informato e basato sui dati. L’integrazione con altre soluzioni IBM, come Watson AI e IBM Cloud, fornisce ulteriori opportunità di ottimizzazione e di trasformazione digitale su larga scala.

\section{Formati di file accettati}
IBM Process Mining supporta diversi formati di file per l'importazione e l'esportazione dei dati, permettendo la flessibilità nell'integrazione con altri strumenti e l'analisi dei processi.

\subsection{CSV (Comma-Separated Values)}
Utilizzato per l'importazione e l'esportazione di dati di origine. È il formato principale per caricare dati da sistemi aziendali e fonti esterne.

\subsection{BPMN 2.0 (Business Process Model and Notation)}
Formato utilizzato per esportare modelli di processo in un linguaggio standard per la modellazione dei processi aziendali. Supporta formati specifici come Bizagi, Bonita, Camunda e IBM Blueworks.

\subsection{XPDL (XML Process Definition Language) 2.1}
Permette di esportare i processi in un formato XML utilizzato comunemente per la definizione di workflow.

\subsection{XES (eXtensible Event Stream)}
Un formato di file standardizzato per rappresentare i log di eventi di processo, utilizzato principalmente per l'analisi dei processi con strumenti di process mining.

\subsection{SVG (Scalable Vector Graphics)}
Utilizzato per l'esportazione di modelli di processo in formato immagine vettoriale, utile per visualizzazioni grafiche di alta qualità dei processi.

\section{Funzionalità}

\subsection{Scoperta dei processi}
IBM Process Mining è in grado di analizzare i log degli eventi provenienti dai sistemi aziendali per generare modelli di processo end-to-end. Questo permette di visualizzare come i processi vengono effettivamente eseguiti, incluse tutte le attività e i percorsi processuali coinvolti.

\subsection{Analisi delle performance}
Lo strumento offre funzionalità per monitorare e valutare la performance dei processi. Questo include l'analisi dei tempi di ciclo, i costi associati a ogni fase del processo, le deviazioni rispetto ai modelli di riferimento e i colli di bottiglia che causano inefficienze operative.

\subsection{Conformance checking}
IBM Process Mining consente di verificare la conformità dei processi rispetto ai modelli di riferimento. Attraverso il confronto dei processi scoperti con i processi teorici o standard, è possibile individuare le deviazioni e correggere le inefficienze.

\subsection{Simulazione dei processi}
La funzionalità di simulazione consente di creare e testare scenari "what-if", simulando modifiche al flusso di lavoro e prevedendo l'impatto di tali modifiche su tempi, costi e risorse. Questa capacità è fondamentale per supportare il processo decisionale in un ambiente basato sui dati.

\subsection{Automazione dei processi}
IBM Process Mining facilita l'automazione dei processi attraverso l'integrazione con strumenti di automazione e RPA (Robotic Process Automation). Identifica i migliori candidati per l'automazione, stima i benefici in termini di costi e performance e genera automaticamente script RPA per automatizzare le attività ripetitive e manuali.

\subsection{Monitoraggio continuo}
È possibile configurare monitoraggi personalizzati per i KPI dei processi, impostare soglie e intervalli di monitoraggio e ricevere notifiche automatiche in caso di anomalie. Questo consente una gestione proattiva dei processi, prevenendo criticità e garantendo il miglioramento continuo.

\subsection{Personalizzazione e dashboard}
IBM Process Mining fornisce dashboard configurabili e interattive che consentono di visualizzare i dati di processo in modo personalizzato. Gli utenti possono filtrare, ordinare e rappresentare i dati per ottenere le informazioni più rilevanti per la loro organizzazione, supportando così l'analisi decisionale.

\section{Implementazioni pratiche: simulazioni}

\end{document}