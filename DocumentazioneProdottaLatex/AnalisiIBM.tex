\documentclass{article}
\usepackage[utf8]{inputenc}
\usepackage[includeheadfoot, margin=1em,headheight=2em]{geometry}
\usepackage{titling}
\geometry{a4paper, left=2cm, right=2cm, top=2cm, bottom=2cm}
\usepackage{graphicx}
\usepackage{float}
\providecommand{\versionnumber}{1.0.0}
\usepackage{enumitem}
\usepackage{array}
\usepackage[italian]{babel}
\newcolumntype{P}[1]{>{\centering\arraybackslash}p{#1}}
\renewcommand{\arraystretch}{1.5} % Default value: 1
\setlength{\droptitle}{-6em}

%font
\usepackage[defaultfam,tabular,lining]{montserrat}
\usepackage[T1]{fontenc}
\renewcommand*\oldstylenums[1]{{\fontfamily{Montserrat-TOsF}\selectfont #1}}

%custom bold 
\usepackage[outline]{contour}
\usepackage{xcolor}
\newcommand{\custombold}{\contour{black}}

%table colors
\usepackage{color, colortbl}
\definecolor{Blue}{rgb}{0.51,0.68,0.79}
\definecolor{LightBlue}{rgb}{0.82,0.87,0.90}
\definecolor{LighterBlue}{rgb}{0.93,0.95,0.96}

%Header
\usepackage{fancyhdr, xcolor}
\pagestyle{fancy}
\let\oldheadrule\headrule% Copy \headrule into \oldheadrule
\renewcommand{\headrule}{\color{Blue}\oldheadrule}% Add colour to \headrule
\renewcommand{\headrulewidth}{0.2em}
\fancyhead[L]{Studio IBM Process Mining}
\fancyhead[C]{Samuele Vignotto}
\fancyhead[R]{\includegraphics[height=1cm]{Logo/Y_LOGO-SOLO.png}}
\setcounter{secnumdepth}{0}

\title{\Huge{\textbf{IBM Process Mining}}\vspace{-1em}}
\author{Samuele Vignotto}
\date{}
\begin{document}
\maketitle
\begin{figure}[h]
  \centering
  \includegraphics[width=6cm, height=6cm]{Logo/Y_LOGO-SOLO.png}
  \label{fig:immagine}
\end{figure}

\newpage
\tableofcontents
\newpage

\section{Introduzione}

IBM Process Mining è uno strumento per l'analisi e l'ottimizzazione dei processi aziendali, progettato per fornire una comprensione di come i processi vengono realmente eseguiti all'interno di un'organizzazione. Grazie alla capacità di sfruttare i dati storici raccolti da diverse fonti aziendali, IBM Process Mining permette di ricostruire il flusso di lavoro effettivo, individuando inefficienze e deviazioni rispetto ai processi ideali.\\
La visualizzazione dettagliata dei processi è uno degli aspetti chiave della soluzione, consentendo agli utenti di vedere come le attività si susseguono nel tempo. Inoltre, lo strumento identifica automaticamente i colli di bottiglia e le ridondanze.\\
Un'altra funzione essenziale di IBM Process Mining è la sua integrazione con tecnologie di automazione e intelligenza artificiale. Questo permette di creare flussi di lavoro automatizzati che semplificano le attività ripetitive, riducendo l'intervento manuale e minimizzando il rischio di errori. Inoltre, lo strumento supporta un monitoraggio continuo delle prestazioni dei processi, consentendo una gestione proattiva e miglioramenti progressivi in tempo reale.\\
Le funzionalità di simulazione offerte da IBM Process Mining permettono di testare vari scenari e prevedere l’impatto di eventuali modifiche, facilitando così un processo decisionale informato e basato sui dati. L’integrazione con altre soluzioni IBM, come Watson AI e IBM Cloud, fornisce ulteriori opportunità di ottimizzazione e di trasformazione digitale su larga scala.

\section{Formati di file accettati}

\section{Fonti di dati supportate}

\section{Funzionalità}

\section{Implementazioni pratiche: simulazioni}

\end{document}