\documentclass{article}
\usepackage[utf8]{inputenc}
\usepackage[includeheadfoot, margin=1em,headheight=2em]{geometry}
\usepackage{titling}
\geometry{a4paper, left=2cm, right=2cm, top=2cm, bottom=2cm}
\usepackage{graphicx}
\providecommand{\versionnumber}{1.0.0}
\usepackage{enumitem}
\usepackage{array}
\usepackage[italian]{babel}
\newcolumntype{P}[1]{>{\centering\arraybackslash}p{#1}}
\renewcommand{\arraystretch}{1.5} % Default value: 1
\setlength{\droptitle}{-6em}

%font
\usepackage[defaultfam,tabular,lining]{montserrat}
\usepackage[T1]{fontenc}
\renewcommand*\oldstylenums[1]{{\fontfamily{Montserrat-TOsF}\selectfont #1}}

%custom bold 
\usepackage[outline]{contour}
\usepackage{xcolor}
\newcommand{\custombold}{\contour{black}}

%table colors
\usepackage{color, colortbl}
\definecolor{Blue}{rgb}{0.51,0.68,0.79}
\definecolor{LightBlue}{rgb}{0.82,0.87,0.90}
\definecolor{LighterBlue}{rgb}{0.93,0.95,0.96}

%Header
\usepackage{fancyhdr, xcolor}
\pagestyle{fancy}
\let\oldheadrule\headrule% Copy \headrule into \oldheadrule
\renewcommand{\headrule}{\color{Blue}\oldheadrule}% Add colour to \headrule
\renewcommand{\headrulewidth}{0.2em}
\fancyhead[L]{Analisi Celonis}
\fancyhead[C]{Samuele Vignotto}
\fancyhead[R]{\includegraphics[height=1cm]{Logo/Y_LOGO-SOLO.png}}
\setcounter{secnumdepth}{0}

\title{\Huge{\textbf{Analisi della piattaforma di process mining Celonis}}\vspace{-1em}}
\author{Samuele Vignotto}
\date{}
\begin{document}
\maketitle
\begin{figure}[h]
  \centering
  \includegraphics[width=6cm, height=6cm]{Logo/Y_LOGO-SOLO.png}
  \label{fig:immagine}
\end{figure}

\newpage
\tableofcontents
\newpage

\section{Introduzione}

Il process mining è una disciplina emergente che combina tecniche di data mining e di analisi dei processi aziendali per fornire una visione approfondita dei processi operativi di un'organizzazione. Una delle piattaforme leader in questo campo è Celonis, una soluzione di process mining avanzata che permette alle aziende di ottimizzare e migliorare continuamente i propri processi aziendali.

\subsection{Estrazione Automatica dei Dati}

Celonis si distingue per la sua capacità di estrarre automaticamente dati da vari sistemi informatici aziendali, tra cui ERP (Enterprise Resource Planning), CRM (Customer Relationship Management) e altri sistemi transazionali. Utilizzando queste informazioni, Celonis crea una rappresentazione visiva e dettagliata dei processi aziendali reali, consentendo agli utenti di identificare inefficienze, colli di bottiglia e deviazioni dai processi standard.

\subsection{Analisi in Tempo Reale}

Una delle funzionalità chiave di Celonis è la sua capacità di fornire un'analisi in tempo reale. Questa caratteristica permette alle organizzazioni di monitorare i propri processi in modo continuo, rilevando e risolvendo problemi immediatamente. Inoltre, Celonis offre potenti strumenti di visualizzazione che permettono agli utenti di esplorare i dati da diverse prospettive, identificando facilmente le aree critiche che necessitano di miglioramenti.

\subsection{Benchmarking}

Un altro aspetto fondamentale di Celonis è la sua funzione di benchmarking, che consente alle aziende di confrontare le proprie prestazioni con quelle di altre organizzazioni o con standard di settore. Questo confronto aiuta a identificare le migliori pratiche e a impostare obiettivi di miglioramento realistici.

\subsection{Integrazione con Machine Learning e Intelligenza Artificiale}

Celonis supporta anche l'integrazione con tecniche di machine learning e intelligenza artificiale, permettendo di prevedere l'andamento dei processi e di suggerire azioni correttive proattive. Questo approccio predittivo è particolarmente utile per anticipare problemi potenziali e mitigare rischi operativi.

\subsection{Automazione dei Processi}

Inoltre, Celonis offre funzionalità di automazione dei processi, che permettono di implementare rapidamente miglioramenti identificati attraverso il process mining. Questo porta a una riduzione dei tempi di ciclo, a una maggiore conformità e a un miglioramento generale dell'efficienza operativa.


\end{document}