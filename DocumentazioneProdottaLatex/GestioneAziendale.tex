\documentclass{article}
\usepackage[utf8]{inputenc}
\usepackage[includeheadfoot, margin=1em,headheight=2em]{geometry}
\usepackage{titling}
\geometry{a4paper, left=2cm, right=2cm, top=2cm, bottom=2cm}
\usepackage{graphicx}
\providecommand{\versionnumber}{1.0.0}
\usepackage{enumitem}
\usepackage{array}
\usepackage[italian]{babel}
\newcolumntype{P}[1]{>{\centering\arraybackslash}p{#1}}
\renewcommand{\arraystretch}{1.5} % Default value: 1
\setlength{\droptitle}{-6em}

%font
\usepackage[defaultfam,tabular,lining]{montserrat}
\usepackage[T1]{fontenc}
\renewcommand*\oldstylenums[1]{{\fontfamily{Montserrat-TOsF}\selectfont #1}}

%custom bold 
\usepackage[outline]{contour}
\usepackage{xcolor}
\newcommand{\custombold}{\contour{black}}

%table colors
\usepackage{color, colortbl}
\definecolor{Blue}{rgb}{0.51,0.68,0.79}
\definecolor{LightBlue}{rgb}{0.82,0.87,0.90}
\definecolor{LighterBlue}{rgb}{0.93,0.95,0.96}

%Header
\usepackage{fancyhdr, xcolor}
\pagestyle{fancy}
\let\oldheadrule\headrule% Copy \headrule into \oldheadrule
\renewcommand{\headrule}{\color{Blue}\oldheadrule}% Add colour to \headrule
\renewcommand{\headrulewidth}{0.2em}
\fancyhead[L]{Studio Gestione Aziendale}
\fancyhead[C]{Samuele Vignotto}
\fancyhead[R]{\includegraphics[height=1cm]{Logo/Y_LOGO-SOLO.png}}
\setcounter{secnumdepth}{0}

\title{\Huge{\textbf{Gestione Aziendale}}\vspace{-1em}}
\author{Samuele Vignotto}
\date{}
\begin{document}
\maketitle
\begin{figure}[h]
  \centering
  \includegraphics[width=6cm, height=6cm]{Logo/Y_LOGO-SOLO.png}
  \label{fig:immagine}
\end{figure}

\newpage
\tableofcontents
\newpage

\section{Introduzione}
La gestione aziendale rappresenta un insieme complesso di attività, strategie e processi finalizzati a garantire l'efficienza e l'efficacia operativa di un'organizzazione. Queste attività includono la pianificazione, l'organizzazione, la direzione e il controllo delle risorse aziendali per raggiungere gli obiettivi prefissati. La gestione aziendale moderna deve affrontare una serie di sfide, tra cui l'adattamento rapido ai cambiamenti di mercato, l'ottimizzazione dei costi operativi e l'incremento della competitività.\\

In questo contesto, il \textit{Process Mining} emerge come una disciplina innovativa che combina principi di data science e gestione dei processi aziendali. Il \textit{Process Mining} utilizza tecniche avanzate di analisi dei dati per estrarre conoscenze utili dai dati generati dai sistemi informativi aziendali, offrendo una visione dettagliata e basata sui fatti delle operazioni aziendali. Questa disciplina si distingue per la sua capacità di analizzare grandi volumi di dati in tempo reale, permettendo alle aziende di monitorare e migliorare continuamente i loro processi.\\

Il \textit{Process Mining} si basa su tre tipi principali di analisi: il \textit{process discovery}, il \textit{conformance checking} e l'\textit{enhancement}. Il \textit{process discovery} implica l'estrazione automatica di modelli di processo dai log degli eventi, offrendo una rappresentazione visiva dei flussi di lavoro reali all'interno dell'organizzazione. Il \textit{conformance checking} confronta questi modelli scoperti con i modelli di processo predefiniti, identificando deviazioni e non conformità che possono indicare inefficienze o aree di rischio. L'\textit{enhancement} si concentra sul miglioramento continuo dei processi, utilizzando le informazioni ottenute per ottimizzare i flussi di lavoro e introdurre innovazioni operative.\\

Uno dei principali vantaggi del \textit{Process Mining} è la sua capacità di fornire una visione trasparente e dettagliata delle operazioni aziendali. Questo livello di trasparenza è fondamentale per identificare colli di bottiglia, ridondanze e altre inefficienze che possono ostacolare la produttività. Inoltre, il \textit{Process Mining} permette di identificare le best practices e diffonderle all'interno dell'organizzazione, promuovendo un miglioramento continuo e una cultura aziendale orientata ai dati.\\

L'implementazione del \textit{Process Mining} richiede un'infrastruttura tecnologica adeguata e una cultura aziendale che valorizzi le decisioni basate sui dati. È essenziale che i dati utilizzati siano di alta qualità, accurati e completi, per garantire che le analisi siano affidabili e possano supportare decisioni strategiche informate. Inoltre, è necessario che il management e il personale siano adeguatamente formati per interpretare e utilizzare i risultati delle analisi di \textit{Process Mining}.\\

In sintesi, il \textit{Process Mining} rappresenta un potente strumento per la gestione aziendale, offrendo una visione basata sui dati dei processi operativi e supportando l'ottimizzazione e l'innovazione continua. Integrando questa disciplina nei propri processi decisionali, le aziende possono ottenere vantaggi competitivi significativi, migliorando l'efficienza, la conformità e la capacità di adattamento alle dinamiche di mercato.\\

\section{Applicazioni del process mining nella gestione aziendale}
Il \textit{Process Mining} trova applicazione in diversi ambiti della gestione aziendale. Un'area di grande impatto è la gestione della supply chain. Analizzando i processi logistici, le aziende possono ottimizzare i tempi di consegna, ridurre i costi operativi e migliorare la soddisfazione del cliente.\\

Un'altra area cruciale è la gestione delle risorse umane. Il \textit{Process Mining} permette di analizzare i processi di assunzione, formazione e gestione delle performance, identificando le fasi che richiedono miglioramenti e garantendo una gestione più efficiente del capitale umano.\\

Nella gestione delle vendite e del marketing, il \textit{Process Mining} aiuta a comprendere meglio il percorso del cliente, ottimizzare le campagne di marketing e migliorare il tasso di conversione. Le informazioni derivate dai log degli eventi offrono una visione dettagliata delle interazioni con i clienti, permettendo strategie di marketing più mirate e personalizzate.\\

\section{Benefici e sfide del process mining}
I benefici del \textit{Process Mining} includono una maggiore trasparenza, l'identificazione rapida delle inefficienze e una migliore capacità decisionale basata sui dati. Tuttavia, l'implementazione del \textit{Process Mining} presenta anche delle sfide. La qualità dei dati è fondamentale: i log degli eventi devono essere accurati e completi per garantire analisi affidabili. Inoltre, è necessaria una cultura aziendale orientata ai dati, dove le decisioni basate sull'evidenza sono valorizzate e supportate a tutti i livelli dell'organizzazione.

\end{document}