\documentclass{article}
\usepackage[utf8]{inputenc}
\usepackage[includeheadfoot, margin=1em,headheight=2em]{geometry}
\usepackage{titling}
\geometry{a4paper, left=2cm, right=2cm, top=2cm, bottom=2cm}
\usepackage{graphicx}
\providecommand{\versionnumber}{1.0.0}
\usepackage{enumitem}
\usepackage{array}
\usepackage[italian]{babel}
\newcolumntype{P}[1]{>{\centering\arraybackslash}p{#1}}
\renewcommand{\arraystretch}{1.5} % Default value: 1
\setlength{\droptitle}{-6em}

%font
\usepackage[defaultfam,tabular,lining]{montserrat}
\usepackage[T1]{fontenc}
\renewcommand*\oldstylenums[1]{{\fontfamily{Montserrat-TOsF}\selectfont #1}}

%custom bold 
\usepackage[outline]{contour}
\usepackage{xcolor}
\newcommand{\custombold}{\contour{black}}

%table colors
\usepackage{color, colortbl}
\definecolor{Blue}{rgb}{0.51,0.68,0.79}
\definecolor{LightBlue}{rgb}{0.82,0.87,0.90}
\definecolor{LighterBlue}{rgb}{0.93,0.95,0.96}

%Header
\usepackage{fancyhdr, xcolor}
\pagestyle{fancy}
\let\oldheadrule\headrule% Copy \headrule into \oldheadrule
\renewcommand{\headrule}{\color{Blue}\oldheadrule}% Add colour to \headrule
\renewcommand{\headrulewidth}{0.2em}
\fancyhead[L]{Studio FreewaySkyline}
\fancyhead[C]{Samuele Vignotto}
\fancyhead[R]{\includegraphics[height=1cm]{Logo/Y_LOGO-SOLO.png}}
\setcounter{secnumdepth}{0}

\title{\Huge{\textbf{FreewaySkyline}}\vspace{-1em}}
\author{Samuele Vignotto}
\date{}
\begin{document}
\maketitle
\begin{figure}[h]
  \centering
  \includegraphics[width=6cm, height=6cm]{Logo/Y_LOGO-SOLO.png}
  \label{fig:immagine}
\end{figure}

\newpage
\tableofcontents
\newpage

\section{Introduzione}
Il software Freeway Skyline prodotto da Eurosystem è un sistema di gestione ERP (Enterprise Resource Planning) progettato per ottimizzare i processi aziendali e ridurre i tempi operativi. É una soluzione multi-azienda e multilingua che permette di standardizzare, semplificare e automatizzare tutti i processi aziendali, fornendo una gestione integrata di tutti i settori dell'impresa, dalla produzione alla contabilità, passando per la logistica e le vendite.

\section{Caratteristiche principali}
\subsection{Flessibilità e modularità}
Freeway Skyline permette alle aziende di personalizzare il software in base alle proprie esigenze specifiche. Questo include moduli per la gestione delle commesse, la pianificazione della produzione, la gestione dei materiali, e la gestione dei rapporti con i clienti.
\subsection{Integrazione e interoperabilità}
Il software può essere integrato con diversi sistemi gestionali e tecnologie aziendali, inclusi database relazionali come Oracle.
\subsection{Settori di applicazione}
Freeway Skyline è adatto a una vasta gamma di settori industriali, inclusi quello manifatturiero, alimentare, chimico, e commerciale. É particolarmente utile per le aziende che operano su commessa, poiché permette di monitorare e ottimizzare ogni fase del processo produttivo.
\subsection{Funzionalità di ricerca avanzata}
Il sistema include Freeway SENSE, una funzionalità di motore di ricerca che velocizza la ricerca di informazioni, dati e documenti all'interno del sistema informativo aziendale.
\subsection{Supporto all'industria 4.0}
Freeway Skyline è progettato per supportare le iniziative di digitalizzazione e automazione tipiche dell'industria 4.0, aiutando le aziende a innovare e migliorare l'efficienza operativa attraverso l'uso di tecnlologie avanzate.

\section{Dati gestiti da Freeway Skyline}

\subsection{Dati di produzione}

\subsubsection{Ordini di Produzione}
Dettagli sugli ordini di produzione, incluse le specifiche dei prodotti, le quantità da produrre, le scadenze e le risorse assegnate.
\subsubsection{Controllo qualità}
Informazioni sui controlli di qualità effettuati durante e dopo la produzione, compresi i risultati dei test e le eventuali azioni correttive.
\subsubsection{Material Requirements Planning}
Dati relativi alla pianificazione dei materiali necessari per la produzione, incluse le previsioni di fabbisogno e le scorte disponibili.

\subsection{Dati di vendita e Customer Relationship Management}
\subsubsection{Gestione preventivi e offerte}
Dettagli sui preventivi e le offerte inviate ai clienti, comprese le condizioni di vendita, i prezzi e i tempi di consegna.
\subsubsection{Anagrafiche clienti}
Informazioni sui clienti, come dettagli di contatto, storico degli ordini, preferenze e attività di marketing.

\subsection{Dati contabili e finanziari}
\subsubsection{Contabilità generale}
Registrazione delle transazioni finanziarie, inclusi i registri contabili, le voci di bilancio e i report finanziari.
\subsubsection{Gestione delle fatture}
Dati relativi alla fatturazione, comprese le fatture emesse e ricevute, i pagamenti e le scadenze.

\subsection{Dati di logistica e magazzino}
\subsubsection{Gestione inventario}
Informazioni sulle scorte di magazzino, compresi i livelli di stock, i movimenti delle merci e le ubicazioni dei prodotti.
\subsubsection{Spedizioni e ricevimenti}
Dettagli sulle spedizioni ai clienti e sulle merci ricevute dai fornitori, compresi i documenti di trasporto e le everifiche di conformità.

\subsection{Dati di risorse umane}
\subsubsection{Anagrafiche dipendenti}
Informazioni sui dipendenti, incluse le informazioni personali, le qualifiche, i contratti di lavoro e le ore lavorate.
\subsubsection{Gestione delle presenze}
Dati sulle presenze e assenze dei dipendenti, turni di lavoro e permessi.

\section{Tipi di operazioni gestite da Freeway Skyline}

\subsection{Contabilità generale e analitica}

Freeway Skyline supporta ampiamente la contabilità generale e analitica, permettendo una gestione precisa dei ratei, risconti, differenze cambio e utili di fine esercizio. Questi strumenti assicurano che tutte le operazioni contabili siano registrate accuratamente, migliorando la trasparenza e facilitando il controllo finanziario. La gestione della contabilità analitica permette di ripartire i costi e i ricavi tra i vari centri di costo e commesse, fornendo una visione dettagliata delle performance aziendali.

\subsection{Gestione delle scadenze e dei flussi di cassa}

La gestione delle scadenze e dei flussi di cassa è una funzionalità fondamentale di Freeway Skyline. Il sistema permette di monitorare in tempo reale le entrate e le uscite di cassa, garantendo che le aziende possano mantenere una gestione finanziaria sana e prevenire problemi di liquidità. Grazie a strumenti avanzati di pianificazione e reportistica, le aziende possono prevedere i flussi di cassa futuri e prendere decisioni strategiche basate su dati accurati.

\subsection{Gestione delle partite aperte e dello scadenzario}

Freeway Skyline facilita la gestione delle partite aperte e dello scadenzario, consentendo alle aziende di tenere traccia di tutte le fatture emesse e ricevute, nonché delle relative scadenze. Questo modulo aiuta a prevenire ritardi nei pagamenti e a mantenere buone relazioni con i fornitori e i clienti. La piattaforma fornisce anche strumenti per il sollecito automatico dei pagamenti, migliorando l'efficienza della gestione finanziaria.

\subsection{Automatizzazione delle operazioni contabili quotidiane}

L'automatizzazione delle operazioni contabili quotidiane è un altro punto di forza di Freeway Skyline. Il sistema permette la rilevazione automatica della ritenuta d'acconto e dei contributi previdenziali, riducendo il carico di lavoro manuale e minimizzando gli errori. Inoltre, le funzionalità di automazione consentono di snellire i processi contabili, migliorando l'efficienza operativa e liberando risorse per attività a maggior valore aggiunto.

\subsection{Gestione delle provvigioni per agenti}

Freeway Skyline offre strumenti avanzati per la gestione delle provvigioni per agenti, inclusa la liquidazione e la contabilizzazione delle fatture. Il sistema permette di definire le regole di calcolo delle provvigioni in base a vari parametri, come il volume delle vendite e i margini di profitto. Grazie a report dettagliati, le aziende possono monitorare le performance degli agenti e ottimizzare le politiche di incentivazione.

\subsection{Calcolo e gestione degli ammortamenti dei cespiti}

Il calcolo e la gestione degli ammortamenti dei cespiti è una funzionalità essenziale di Freeway Skyline. Il sistema consente di gestire l'intero ciclo di vita dei cespiti, dall'acquisizione alla dismissione, calcolando automaticamente gli ammortamenti secondo le normative vigenti. Questa gestione accurata degli ammortamenti permette alle aziende di mantenere un controllo rigoroso sui propri beni e di ottimizzare le decisioni di investimento.

\subsection{Emissione e gestione dei bilanci aziendali}

Freeway Skyline supporta l'emissione e la gestione dei bilanci aziendali, offrendo strumenti per la preparazione, l'elaborazione e l'analisi dei bilanci. Il sistema consente di generare bilanci in forma sintetica o dettagliata, riclassificati secondo le normative italiane e internazionali. Le funzionalità avanzate includono l'analisi degli scostamenti e la reportistica dinamica, permettendo alle aziende di monitorare la propria situazione finanziaria e di prendere decisioni strategiche basate su dati solidi.

\subsection{Gestione integrata delle relazioni con i clienti (CRM)}

La gestione integrata delle relazioni con i clienti è facilitata dal modulo CRM di Freeway Skyline. Questo modulo permette alle aziende di gestire efficacemente le anagrafiche dei clienti, tracciare le interazioni e gestire le opportunità di vendita. Il sistema offre strumenti per la pianificazione delle attività commerciali e la gestione delle campagne di marketing, migliorando la soddisfazione e la fidelizzazione dei clienti.

\subsection{Supporto alle comunicazioni telematiche}

Freeway Skyline offre supporto completo per le comunicazioni telematiche, come spesometro, liquidazione IVA periodica e dichiarazioni di intento. Queste funzionalità garantiscono che le aziende possano rispettare le normative fiscali vigenti, automatizzando i processi di generazione e invio delle comunicazioni obbligatorie. Il sistema facilita la conformità normativa e riduce il rischio di errori e sanzioni.

\subsection{Gestione del ciclo attivo e passivo delle vendite e degli acquisti}

La gestione del ciclo attivo e passivo delle vendite e degli acquisti è una componente chiave di Freeway Skyline. Il sistema consente di pianificare e monitorare tutte le fasi del ciclo di vendita e di approvvigionamento, dalla gestione degli ordini alla pianificazione delle spedizioni e al controllo qualità fornitori. Le funzionalità avanzate di reportistica e analisi aiutano le aziende a ottimizzare i processi di vendita e di acquisto, migliorando l'efficienza operativa e la soddisfazione dei clienti.


\section{Contabilità generale in Freeway Skyline}

\subsection{Parametrizzazione dei conti e delle causali contabili}

Freeway Skyline permette una completa parametrizzazione dei conti e delle causali contabili, offrendo la flessibilità necessaria per adattarsi alle specifiche esigenze contabili di ogni azienda. Questa funzionalità consente di definire e configurare i conti e le causali in modo preciso, garantendo che tutte le transazioni siano registrate correttamente e secondo le norme vigenti. La parametrizzazione avanzata facilita la gestione delle operazioni contabili, migliorando l'accuratezza e la coerenza dei dati finanziari.

\subsection{Gestione multiditta e multibanca}

Una delle caratteristiche distintive di Freeway Skyline è la capacità di gestire ambienti multiditta e multibanca. Questo modulo consente di amministrare simultaneamente più entità legali e rapporti bancari all'interno di un'unica piattaforma. Le aziende possono consolidare le operazioni contabili e finanziarie di diverse ditte, migliorando l'efficienza e la trasparenza. Inoltre, la gestione multibanca permette di monitorare e riconciliare le transazioni bancarie provenienti da diversi istituti, assicurando un controllo accurato dei flussi di cassa.

\subsection{Monitoraggio e riconciliazione dei movimenti bancari}

Freeway Skyline offre strumenti avanzati per il monitoraggio e la riconciliazione dei movimenti bancari. Questa funzionalità permette di tracciare tutte le transazioni bancarie, garantendo che ogni movimento sia registrato correttamente nel sistema contabile. La riconciliazione automatizzata confronta i movimenti bancari con le registrazioni contabili, identificando rapidamente eventuali discrepanze. Questo processo migliora l'accuratezza dei dati finanziari e riduce il rischio di errori, facilitando una gestione finanziaria più efficiente.

\subsection{Gestione delle scadenze di pagamento e solleciti}

La gestione delle scadenze di pagamento e dei solleciti è un aspetto cruciale per mantenere una buona salute finanziaria. Freeway Skyline consente di monitorare tutte le scadenze di pagamento, sia per le fatture emesse che per quelle ricevute. Il sistema invia notifiche automatiche per ricordare le scadenze imminenti, aiutando a evitare ritardi nei pagamenti. Inoltre, offre funzionalità di sollecito automatico per i crediti in sospeso, migliorando il flusso di cassa e riducendo i tempi di recupero dei crediti.

\subsection{Emissione di bilanci di verifica e bilanci competenziati infrannuali}

Freeway Skyline supporta l'emissione di bilanci di verifica e bilanci competenziati infrannuali, strumenti essenziali per il monitoraggio continuo delle performance finanziarie. I bilanci di verifica forniscono una fotografia della situazione contabile in un determinato momento, aiutando a identificare eventuali errori o incongruenze. I bilanci competenziati infrannuali, invece, permettono di valutare periodicamente la situazione economica e finanziaria, offrendo una visione più dettagliata e tempestiva delle performance aziendali.

\subsection{Generazione di report e analisi dettagliate tramite Freeway Business Intelligence}

La piattaforma Freeway Skyline integra Freeway Business Intelligence, un potente strumento per la generazione di report e analisi dettagliate. Questo modulo permette di creare report personalizzati e di analizzare i dati finanziari in profondità, supportando decisioni strategiche basate su informazioni accurate e aggiornate. Le funzionalità di Business Intelligence offrono visualizzazioni intuitive dei dati, consentendo di individuare trend, monitorare le performance e prevedere scenari futuri, migliorando la capacità decisionale dell'azienda.


\section{Contabilità analitica in Freeway Skyline}

\subsection{Ripartizione dei valori su centri di costo/ricavo e voci analitiche}

Freeway Skyline offre una gestione avanzata della contabilità analitica, permettendo la ripartizione dei valori su centri di costo e di ricavo, nonché su specifiche voci analitiche. Questo consente alle aziende di tracciare e analizzare con precisione i costi e i ricavi associati a differenti dipartimenti, progetti o linee di prodotto. La ripartizione analitica dei valori facilita il controllo di gestione e permette una valutazione dettagliata della redditività e dell'efficienza operativa.

\subsection{Integrazione con il piano dei conti}

La contabilità analitica di Freeway Skyline è strettamente integrata con il piano dei conti, garantendo una coerenza completa tra le registrazioni contabili e le analisi analitiche. Questa integrazione permette di mantenere un allineamento costante tra la contabilità generale e quella analitica, assicurando che tutte le transazioni siano riflesse accuratamente nei report finanziari. Inoltre, facilita la riconciliazione e il consolidamento dei dati contabili, migliorando la trasparenza e l'affidabilità delle informazioni finanziarie.

\subsection{Gestione del budget}

La gestione del budget è una funzionalità chiave della contabilità analitica di Freeway Skyline. Il sistema permette di costruire budget basati su valori storici o di importarli da sistemi esterni, offrendo una flessibilità completa nella pianificazione finanziaria. La gestione del budget consente alle aziende di stabilire obiettivi finanziari, monitorare le performance rispetto ai budget previsti e apportare eventuali correzioni per raggiungere gli obiettivi prefissati. Questo strumento è essenziale per un controllo finanziario efficace e per la gestione strategica delle risorse.

\subsection{Analisi degli scostamenti e reportistica specifica}

Freeway Skyline supporta l'analisi dettagliata degli scostamenti tra i valori reali e quelli del budget, fornendo reportistica specifica per individuare le aree di miglioramento. Questo modulo consente di identificare rapidamente le deviazioni dai piani finanziari e di analizzare le cause degli scostamenti. La reportistica avanzata offre una visione chiara e immediata delle performance aziendali, supportando decisioni informate e tempestive. L'analisi degli scostamenti è fondamentale per mantenere il controllo sui costi e per ottimizzare la gestione finanziaria.

\subsection{Integrazione con i processi gestionali e contabili aziendali}

La contabilità analitica di Freeway Skyline è completamente integrata con i processi gestionali e contabili aziendali, assicurando un flusso continuo di informazioni tra i vari dipartimenti. Questa integrazione facilita la condivisione dei dati, riduce la duplicazione delle informazioni e migliora l'efficienza operativa. Le aziende possono beneficiare di una visione unificata delle operazioni finanziarie e gestionali, permettendo una gestione più coordinata e strategica delle risorse.

\subsection{Supporto per la gestione tradizionale e per l'analisi ABC (Activity Based Costing)}

Freeway Skyline offre supporto sia per la gestione contabile tradizionale che per l'analisi ABC. L'analisi ABC permette di allocare i costi in base alle attività effettivamente svolte, offrendo una visione più precisa dei costi associati a prodotti, servizi e clienti. Questo approccio consente alle aziende di identificare le attività a maggior valore aggiunto e di ottimizzare l'allocazione delle risorse. Il supporto per entrambe le metodologie garantisce una flessibilità completa nella gestione dei costi e nella pianificazione strategica.


\section{Madalità di salvataggio dei dati}
Il software Freeway Skyline utilizza un database relazionale, integrato con Oracle.\\
In Freeway Skyline, i dati sono organizzati in tabelle strutturate, facilmente interrogabili e gestibili tramite SQL (Structured Query Language). Oracle, come database principale supportato, garantisce elevati standard di sicurezza e prestazioni. Questo permette alle aziende di gestire i propri dati in modo centralizzato, fornendo agli utenti l'accesso a informazioni specifiche attraverso un portale di navigazione con vista profilata, in base ai loro ruoli.\\

\end{document}
